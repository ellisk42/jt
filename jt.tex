\documentclass{beamer}
\usepackage{tipa}
\usepackage{phonrule}
%\setbeamersize{text margin left=10pt,text margin right=10pt}
\usetheme{metropolis}
\usepackage{listings}
\usepackage{amsfonts}
\usepackage{dsfont}
\usepackage{amsmath}
\usepackage{bbm}
\usepackage{verbatim}
\usepackage{phonrule}
%\usepackage{tipa}
\usepackage{tikz}
\usetikzlibrary{fit}
\usepackage{color}
\usepackage{booktabs}
\usepackage{tipa}
\usepackage{amssymb}
\usepackage{verbatim}
\usepackage[absolute,overlay]{textpos}
\usepackage{pifont}% http://ctan.org/pkg/pifont
\usepackage{caption}
\usepackage{subcaption}
\newcommand{\cmark}{\ding{51}}%
\newcommand{\xmark}{\ding{55}}%
\usetikzlibrary{bayesnet}
\usetikzlibrary{decorations.markings}
\usetikzlibrary{decorations.pathmorphing}
\tikzset{squiggle/.style={decorate, decoration={snake,amplitude=.4mm}}}
\usepackage{xcolor}
\definecolor{pop1}{HTML}{1F78b4}
\definecolor{pop2}{HTML}{164C13}
\definecolor{pop3}{HTML}{d95F02}
\definecolor{orange}{HTML}{d95F02}
\definecolor{teal}{HTML}{1b9e77}
\newcommand{\pop}[1]{\textcolor{pop1}{#1}}
\newcommand{\popp}[1]{\textcolor{pop2}{#1}}
\newcommand{\tree}[1]{\textcolor{pop3}{#1}}
\newcommand{\orange}[1]{\textcolor{orange}{#1}}
\newcommand{\teal}[1]{\textcolor{teal}{#1}}
\newcommand{\code}[1]{{\footnotesize\texttt{#1}}}
\newcommand{\greenCode}[1]{{\footnotesize\popp{\code{#1}}}}
\newcommand{\blueCode}[1]{{\footnotesize\pop{\code{#1}}}}
\definecolor{backgroundGreen}{HTML}{23373b}
\lstset{escapeinside={<@}{@>}}
\usepackage{pgf}  

%% \usepackage[sfdefault]{FiraSans} %% option 'sfdefault' activates Fira Sans as the default text font
%% \usepackage[T1]{fontenc}
%% \renewcommand*\oldstylenums[1]{{\firaoldstyle #1}}


\newcommand{\Expect}{\mathds{E}} %{{\rm I\kern-.3em E}}
\newcommand{\indicator}{\mathds{1}} %{{\rm I\kern-.3em E}}
\newcommand{\expect}{\mathds{E}} %{{\rm I\kern-.3em E}}
\newcommand{\probability}{\mathds{P}} %{{\rm I\kern-.3em P}}
\DeclareMathOperator*{\argmin}{arg\,min} % thin space, limits underneath in displays
\DeclareMathOperator*{\argmax}{arg\,max} % thin space, limits underneath in displays

\usepackage[absolute,overlay]{textpos}

\newcommand{\nextForm}[1]{\rotatebox[origin=c]{270}{$_{\curvearrowright}$}$_{#1}$}
 
\usepackage{amsfonts}
\usepackage{tabularx}
%\usepackage{color}
\usepackage{graphicx}
\usepackage{booktabs}
\usepackage{xcolor}
\usepackage{tikz}
\usetikzlibrary{trees}
\usetikzlibrary{fit}
\usetikzlibrary{calc}
\usetikzlibrary{bayesnet}
\usepackage[absolute,overlay]{textpos}
\usepackage{stmaryrd}
\newcommand{\sem}[1]{\llbracket #1\rrbracket}
\newcommand{\tuple}[1]{\ensuremath{\left \langle #1\right \rangle}}
\newcommand{\messageOverlay}[1]{
      \tikz[overlay,remember picture]
      \node[align=left,fill=backgroundGreen,text=white] at (current page.center){#1};
}
\usepackage{booktabs}
\usepackage{tipa}
\usepackage{amssymb}
\usepackage{verbatim}
\usepackage[absolute,overlay]{textpos}
\usepackage{pifont}% http://ctan.org/pkg/pifont
\newcommand{\cmark}{\ding{51}}%
\newcommand{\xmark}{\ding{55}}%
\usetikzlibrary{bayesnet}
\usetikzlibrary{decorations.markings}

\newcommand\Wider[2][3em]{%
\makebox[\linewidth][c]{%
  \begin{minipage}{\dimexpr\textwidth+#1\relax}
  \raggedright#2
  \end{minipage}%
  }%
}

\newcommand{\denotation}[1]{{\llbracket #1 \rrbracket}}

\usepackage[utf8]{inputenc}
\newcommand{\reduce}{\longrightarrow}
\usepackage{amssymb}% http://ctan.org/pkg/amssymb
\usepackage{pifont}% http://ctan.org/pkg/pifont

\usepackage{fancyvrb}

\usepackage[most]{tcolorbox}
\definecolor{block-gray}{gray}{0.10}
\newtcolorbox{mycode}{colback=block-gray,grow to right by=0mm,grow to left
by=0mm, boxrule=0pt,boxsep=0pt,breakable,fontupper=\color{white}}

%% Program ::=
%%   (if Bool List
%%     (append RecursiveList
%%             RecursiveList
%%             RecursiveList))
%% RecursiveList ::= List
%%          | (recurse List)

            

\usepackage{arydshln}

\newcommand{\Expect}{\mathds{E}} %{{\rm I\kern-.3em E}}
\newcommand{\Probability}{\mathds{P}} %{{\rm I\kern-.3em P}}

\DeclareMathOperator*{\argmax}{arg\,max}
%Information to be included in the title page:
\title{Building Machines that Discover Generalizable, Interpretable Knowledge}%,\\ by learning to write code that writes code}
\author{Kevin Ellis}
\institute{MIT} 
\date{2020}
  
 
\begin{document}
 
\frame{\titlepage}

\begin{frame}{What computational problems are solved by intelligence?}
  \Wider[5em]{
    \centering\textbf{an endless range of problems}

    \begin{tabular}{ccc}
      language&science&design\\%&engineering\\
      \includegraphics[height = 2cm]{figures/acquisition}&
      \includegraphics[height = 2cm]{figures/blackboard}&
      \includegraphics[height = 2cm]{figures/textile2}\\
%      \includegraphics[height = 2cm]{figures/primitiveTools}\\
      using new devices&writing new characters&coding\\
      \includegraphics[height = 2cm]{figures/noodles}&
      \includegraphics[height = 2cm]{figures/Korean}&
      \includegraphics[height = 2cm]{../ecPaper/figures/1975.png}\\
      \multicolumn{2}{c}{engineering}&play\\
      \includegraphics[height = 2cm]{figures/igloo}&
      \includegraphics[height = 2cm]{../ecPaper/figures/aqueduct}&
      \includegraphics[height = 2cm]{figures/childPlaying}
    \end{tabular}
  }
  %% \pause
  %% \messageOverlay{- learning from modest data/experience\\
  %%   - communicating \& representing knowledge in understandable formats\\
  %%   - bootstrapping \& learning-to-learn\\
  %%   - creatively composing knowledge to produce new concepts and artifacts\\
  %%   \phantom{test}(\& compositionality)
  %% }
\end{frame}

\begin{frame}{What computational frameworks can \\contribute to this picture?}

  \Wider[5em]{\begin{center}
      \begin{tabular}{ccc}
        \multicolumn{3}{c}{Three AI traditions}\\\\
        \visible<2->{Symbolic}&\visible<3->{Probabilistic}&\visible<4->{Neural}\\
        \visible<2->{\begin{tabular}{c}
          \includegraphics[width = 3cm]{figures/deepBlue}\\
          \includegraphics[width = 3cm]{figures/Mathematica}
        \end{tabular}}&
        \visible<3->{\begin{tabular}{c}
          \includegraphics[width = 3cm]{figures/bayesianNetwork}\\
          \includegraphics[width = 3cm]{figures/seismic}
        \end{tabular}}&
        \visible<4->{\begin{tabular}{c}
          \includegraphics[width = 3cm]{figures/dqn2}\\
          \includegraphics[width = 3cm]{figures/Alex2}
        \end{tabular}}
      \end{tabular}
  \end{center}}
  
  

  \visible<5->{
    \messageOverlay{\textbf{Program induction}\\ %% roots in the works of Solomonoff, Fodor, McCarthy.
    machines that learn, perceive, and reason,\\\phantom{test} by writing their own code
    }
  }

    %% strong generalization

    %% data efficiency

    %% interpretability

    %% compositional reuse

    %% universal Turing-computable expressiveness
    %Combines and complements these three traditions
%  }

  %% \pause

  %% \messageOverlay{my work:\\%% \\\phantom{test} how program induction can contribute to the engineering of machine intelligence\\
  %%   %% in engineering terms:\\
  %%   \phantom{test} how learning can make program synthesis more scalable \& practical\\
  %%   \phantom{test} how program synthesis can contribute to machine intelligence
  %%  } 
  
  

\end{frame}

\begin{frame}{Why program induction?}
  \Wider[5em]{  \begin{tabular}{ccc}
    \begin{tabular}{l}
      strong generalization\\
      +data efficiency
    \end{tabular}&
    \begin{tabular}{l}
      interpretability\\
    \end{tabular}
    &\begin{tabular}{l}
       universal expressivity\\
     \end{tabular}\\
    \includegraphics[width = 3cm]{figures/polynomialExtrapolation}&
    \includegraphics[width = 3.5cm]{figures/explodedCAD}&
    \includegraphics[width = 4cm]{figures/turingMachine}
  \end{tabular}}
\end{frame}

\begin{frame}{Why didn't this old idea work?}
  \Wider[4em]{  Program induction goes back to the 1956 Dartmouth Workshop that founded the field of AI }

  \vspace{1cm}

  \Wider[4em]{
    \begin{tabular}{ccc}
      \includegraphics[height = 2.5cm]{figures/Dartmouth}&
      \includegraphics[height = 2.5cm]{figures/Dartmouth2}&
      \includegraphics[height = 2.5cm]{figures/Dartmouth3}
    \end{tabular}
  }
\end{frame}

\begin{frame}{Why will it work this time?}

  better toolkits:
  \begin{itemize}
  \visible<2->{\item \textbf{probabilistic} methods for uncertainty and learning-to-learn}
  \visible<3->{\item \textbf{neural} methods for guiding combinatorial search}
  \visible<4->{\item \textbf{symbolic} methods, from the \textbf{programming languages} community
    \begin{itemize}
    \item maturing \textbf{program synthesis} techniques
    \item type systems, program analysis, constraint solving, ...
    \end{itemize}   }
  \end{itemize}

  \visible<5->{better problems:
  \begin{itemize}
  \item inverse CAD [Kulkarni et al. 2015]
%  \item probabilistic program synthesis [???]
  \item synthesizing human-understandable models [Evans et al. 2019]
  \item natural language$\to $code [Liang et al. 2011; Zettlemoyer et al. 2007]
  \item programming by examples [Gulwani 2011], computer-aided-programming [Solar-Lezama 2008]
  \end{itemize}} 

\end{frame}




\begin{frame}[fragile]{Perception, Synthesizing models, Learning-to-Learn}

  Theme: high-level visual understanding, pixels$\to $programs %going from high-dimensional percepts to symbolic program representations

  \only<1>{\includegraphics[width = \textwidth]{graphicsTeaser.pdf}}
  \only<2->{{\centering\includegraphics[width = 0.6\textwidth]{assets/we_are_cool_4.png}}}
  
  \only<2>{Theme: Synthesizing human-understandable models}
  \only<2>{{\centering\phantom{this is a test}\includegraphics[width = 0.6\textwidth]{phonology/qualitativeSupplement2.pdf}}}
  \only<3->{\vspace{-1cm}Theme \#2: Synthesizing interpretable models\includegraphics[width = 0.15\textwidth]{phonology/qualitativeSupplement2.pdf}}
  
  \only<3->{Theme: Learning to synthesize programs}
  \only<3>{\includegraphics[width = 0.9\textwidth]{statement/taskbar2.png}}

%  \only<3->{\includegraphics[width = 1\textwidth]{statement/taskbar_short.png}}

\end{frame}

\begin{frame}{}
  \begin{center}
    \begin{tabular}{l}
      {\textcolor{black}{Program Induction and perception}}\\
      \phantom{Program Induction and }{\textcolor{gray}{learning to learn}}\\
      \phantom{Program Induction and }{\textcolor{gray}{interpretable models}}
      \end{tabular}
  \end{center}
\end{frame}

\begin{frame}{High-level, abstract visual abilities}

  %% \begin{textblock*}{1cm}(1cm,0.25\textwidth)
  %%   \begin{exampleblock}{}
  %%     test
  %%     \end{exampleblock}
  %% \end{textblock*}


  \begin{tabular}{ccc}
    \begin{tabular}{c}
      ...in art
      \end{tabular}&
    \begin{tabular}{c}
      \includegraphics[width = 0.3\textwidth]{figures/textile}
      \end{tabular}&
    \begin{tabular}{c}
      \includegraphics[width = 0.3\textwidth]{figures/mce}
      \end{tabular}\\
%    \visible<3->{\includegraphics[width = 0.3\textwidth]{figures/dinner}}&
    \begin{tabular}{c}
      ...in engineering
      \end{tabular}&
    \begin{tabular}{c}
      \includegraphics[width = 0.3\textwidth]{figures/aqueduct}
    \end{tabular}&
%    \visible<4->{\hspace*{-1cm}\includegraphics[width = 0.4\textwidth]{figures/bg.png}}&
    \begin{tabular}{c}
      \includegraphics[width = 0.3\textwidth]{figures/CAD_laptop}
    \end{tabular}\\
    \begin{tabular}{c}
      ...in AI
    \end{tabular}&
    \begin{tabular}{c}
      \includegraphics[width = 0.3\textwidth]{figures/rnn}
    \end{tabular}&
    \begin{tabular}{c}
      \includegraphics[width = 0.3\textwidth]{figures/graphicalModel}
      \end{tabular}
    
 %   \visible<6->{\includegraphics[width = 0.3\textwidth]{figures/fractal_tree}}&
  \end{tabular}

  \visible<2->{
    \messageOverlay{why?\\\phantom{te}impute missing objects, extrapolate percepts,\\\phantom{te}learn visual concepts (`arch', `spiral', `Ising model'),\\\phantom{te}assist graphic design, assist 3D modeling\\how?\\\phantom{te}machine learning + \\\phantom{te}program synthesis techniques from programming languages community}
  }
\end{frame}

\begin{frame}[fragile]{Learning to infer graphics programs from hand-drawn images}
\hspace*{-1cm}\begin{tikzpicture}
    \node(picture1) at (0,-1) {\includegraphics[height = 2.2cm]{TikZfigures/expert-31.png}};
    \node[draw,right=1cm of picture1] (c1) {
        \begin{lstlisting}[basicstyle = \small\ttfamily]
for (0 <= i < 3)
 rectangle(3*i,-2*i+4,
           3*i+2,6)
 for (0 <= j < i + 1)
  circle(3*i+1,-2*j+5)
        \end{lstlisting}
      };
    
    \draw[very thick,->] (picture1.east)  -- (c1.west);

    \node(l1)[anchor=south] at ([yshift=0.75cm,xshift=2.5cm]picture1.north) {model infers program from drawing};
\end{tikzpicture}

    \vskip0pt plus 1filll
    \hspace*{-0.75cm}Ellis, Ritchie, Solar-Lezama, Tenenbaum. NeurIPS 2018.
\end{frame}

\begin{frame}[fragile]{Learning to infer graphics programs from hand-drawn images}
  
\hspace*{-1cm}\begin{tikzpicture}
    \node(picture1) at (0,-1) {\includegraphics[height = 2.2cm]{TikZfigures/expert-31.png}};
    \node[draw,right=1cm of picture1] (c1) {
        \begin{lstlisting}[basicstyle = \small\ttfamily]
for (0 <= i < 3 <@\textcolor{red}{\textbf{+ 1}}@>)
 rectangle(3*i,-2*i+4,
           3*i+2,6)
 for (0 <= j < i + 1)
  circle(3*i+1,-2*j+5)
        \end{lstlisting}
      };
    
    \draw[very thick,->] (picture1.east)  -- (c1.west);    

    \node(e)[anchor=west] at ([xshift=1cm]c1.east) {\includegraphics[width = 2.2cm]{TikZfigures/31-extrapolated.png}};
    
    \draw[very thick,->] (c1.east)  -- (e.west);
    \node(l1)[anchor=south] at ([yshift=0.75cm,xshift=2.5cm]picture1.north) {model infers program from drawing};
    \node[anchor=north west] at ([yshift=0.09cm]l1.south west) {\textbf{zero-shot generalization / extrapolation}};
\end{tikzpicture}

\visible<2>{

  \phantom{te}
  
  \hspace*{-1cm}\includegraphics[width = 1.2\textwidth]{TikZfigures/extrapolationMatrix1.png}
}

%    \vskip0pt plus 1filll
    \hspace*{-0.75cm}Ellis, Ritchie, Solar-Lezama, Tenenbaum. NeurIPS 2018.
\end{frame}

\begin{frame}[fragile]{How to infer graphics programs from hand-drawn images}

\hspace*{-1cm}\begin{tikzpicture}[scale=0.9]
  \node[ thick,anchor = west,inner sep=0pt,label={[yshift = 0.3cm]{\small \begin{tabular}{c}
          \textbf{Image}\\
          \textbf{(Observed)}
  \end{tabular}}}](observation) at (0,0) {\includegraphics[width = 1.5cm]{TikZfigures/expert-39-trimmed.png}};
    \node[ultra thick,anchor = west,inner sep=0pt](traceSource) at (3.7,0.5){    \begin{lstlisting}[basicstyle = \scriptsize\ttfamily]
line, line,
rectangle,
line, ...
\end{lstlisting}};
    \node[ultra thick,anchor = west,inner sep=0pt](traceImage) at (4.15,-0.6) {
      \includegraphics[width = 0.9cm]{TikZfigures/39-parse.png}}; 
    \node(trace)[draw,thin,fit = (traceImage) (traceSource), label = above:{{\small \begin{tabular}{c}
            \textbf{Drawing Commands}\\
            \textbf{(Latent)}
    \end{tabular}}}] {};
    
    \node[draw, thick,anchor = west,inner sep=2pt,label=above:{\small \begin{tabular}{c}
          \textbf{Program}\\
          \textbf{(Latent)}
    \end{tabular}}](program) at (7.7,0) {
      \begin{lstlisting}[basicstyle = \scriptsize\ttfamily]
for (j < 3)
for (i < 3)
if (...)
 line(...)
 line(...)
rectangle(...)
    \end{lstlisting}};

%    \node[ultra thick,anchor = west,inner sep=0pt,label=below:{\small Similarity}](similarity) at (11.3,0) {\includegraphics[width = 1cm]{TikZfigures/expert-38-trim.png}};


    
    \draw[->, ultra thick] ([yshift=10]trace.west) to[out = 150,in = 30] node[midway,yshift = 6]{{\small render}} ([xshift=5,yshift=10]observation.east); % -- node[fill = white,rotate = -90] {{\small Rendering}}
    \draw[->, ultra thick] ([yshift=10]program.west) to[out = 150,in = 30] node[midway,yshift = 6] {{\small exec}} ([yshift=10]trace.east);
    
    \pause
    \draw[->, very thick, red] ([yshift = -15,xshift=3]observation.east) to[out = -30,in = -150] node[midway,yshift = -19,xshift=0]{{\small \begin{tabular}{l}
          learning+ \\stochastic search
    \end{tabular}}} ([yshift = -15]trace.west);

    \draw[->, very thick, red] ([yshift = -15,xshift=0]trace.east) to[out = -30,in = -150] node[midway,yshift = -25,xshift=18]{{\small \begin{tabular}{l}
          learning+ \\program synthesis
    \end{tabular}}} ([yshift = -15]program.west);
%    \draw[->, thick, red, very thick] ([xshift = 10]trace.south) to[out = -10,in = -170] node[midway,yshift = -6]{{\small Learning + Program synthesis}} ([xshift = 10]program.south);
%    \draw[->, thick, red] (program.east) to[out = 80,in = 180] (errors.west);
%    \draw[->, thick] (program.east) to[out = 40,in = -230] (similarity.west);


    \draw[decoration = {brace,mirror,raise = 5pt},decorate,thick]
    ([yshift = -40,xshift = -130]trace.south) -- node[below = 6pt] {{\small  Image$\to$Parse}} ([yshift = -40,xshift = -5]trace.south);
    \draw[decoration = {brace,mirror,raise = 5pt},decorate,thick]
    ([xshift = 5,yshift = -40]trace.south) -- node[below = 6pt] {{\small Parse$\to$Program}} ([xshift = 170,yshift = -40]trace.south);

    \pause
    \node[ultra thick,anchor = west,inner sep=0pt,label=below:{\small extrapolation}](extrapolate) at (11.8,0.9) {\includegraphics[width = 1.3cm]{TikZfigures/39-extrapolated.png}};
    \draw[->, thick, red, very thick] (program.east) to[out = 60,in = 180] (extrapolate.west);
    \node[ultra thick,anchor = west,text width = 2.3cm,inner sep=0pt](errors) at (11.1,-1) {\small error correction};
    
    \draw[decoration = {brace,mirror,raise = 5pt},decorate,thick]
([xshift = 180,yshift = -40]trace.south) -- node[below = 6pt] {{\small Applications}} ([xshift = 255,yshift = -40]trace.south);
  \end{tikzpicture}
\end{frame}

%% \begin{frame}{Parsing images into specs (\LaTeX~TikZ commands)}

%% {\footnotesize  Neurally Guided Procedural Modeling (Ritchie et al 2016) + \\Attend, Infer, Repeat (Eslami et al 2016)}
%%   \vspace{-0.5cm}
%%   \begin{center}
%%     \begin{tabular}{c}
%% \raisebox{-.5\height}{    \includegraphics[width = 0.8\textwidth]{../TikZpaper/architecture.pdf}            }
%%       \end{tabular}

%%   \end{center}

%%     \newcommand{\noisySize}{0.12\textwidth}
%%     \visible<2->{
%% \hspace{0.1\textwidth}      \begin{minipage}[t]{\noisySize}
%%         \centering\includegraphics[width = \textwidth]{TikZfigures/expert-60-reduced.png}\\
%%                                   {\small     hand drawing}
%%       \end{minipage}\hspace{0.05\textwidth}%
%%       \begin{minipage}[t]{\noisySize}
%%         \centering\includegraphics[width = \textwidth]{TikZfigures/60-1-reduced.png}\\
%%                                   {\small noisy \\rerender}
%%       \end{minipage}}%
%%     %% \visible<3->{\begin{minipage}{0.35\textwidth}
%%     %%     \begin{flushright}
%%     %%       \textbf{Learned distance metric}\\
%%     %%       \small    Serves as likelihood surrogate
%%     %%       \begin{align*}
%%     %%         \displaystyle\qquad    -\log L_{\text{learned}}(\text{render}&(S_1)|\text{render}(S_2))\approx\\& |S_1 - S_2| + |S_2 - S_1|
%%     %%       \end{align*}
%%     %%       $S_1$: noisy render\\
%%     %%       $S_2$: nonnoisy render
%%     %%     \end{flushright}
%%     %% \end{minipage}}%
%%     \hspace{0.2\textwidth}%
%%     \only<3>{\begin{minipage}[c]{0.3\textwidth}
%% \vspace{0.4cm}      \includegraphics[width = 0.8\textwidth]{TikZfigures/evaluationPhase1}
%%     \end{minipage}}
%%         \only<4>{\begin{minipage}[c]{0.3\textwidth}
%% \vspace{0.4cm}      \includegraphics[width = 0.8\textwidth]{TikZfigures/evaluationPhase2}
%%     \end{minipage}}
%% \end{frame}

%% \begin{frame}{Learning to quickly synthesize programs}
%%     Learn search policy $\pi (\text{program subspace} | \text{spec})$
%%   \\Think of the subspace as an ``ansatz''

%% OBJECTIVE: Small subspace for tractability while also being likely to contain good programs

%% %  \vspace{0.7cm}
%% \Wider[5em]{   \begin{tikzpicture}[scale=0.7]
%%     \footnotesize
%%     \node[anchor = west] at (0,5.25) {{ Entire program search space}};
%%     %\footnotesize
%%     \draw[fill = yellow,fill opacity = 0.15,draw = black] (0,0) rectangle (5,5);
%% \pause
%% \draw [fill = yellow, opacity = 0.2] (0,0)--(0,2)--(4,0)--(0,0);
%%     \draw (0,2) -- node[below,sloped]{short progs} (4,0);
%%     \draw (0,2) -- node[above,sloped]{long progs} (4,0);
%%     \pause
    

%%     \draw [fill = yellow, opacity = 0.6] (2.5,5)--(3.2,0)--(5,0)--(5,5);
%%     \draw (2.5,5) -- node[above,sloped]{progs w/ reflections} (3.2,0);
%%     \pause
    
%%     \draw [fill = yellow, opacity = 0.4] (0,5)--(0,1)--(3,5);
%%     \draw (0,1) -- node[above,sloped]{ progs w/ loops} (3,5);
%%     \pause
    
%%     \node(p1)[anchor =west ] at (5.0,3.5) {$\pi(\text{short, no loop/reflect}|S) = $};
%%     \draw [fill = yellow, fill opacity = 0.2,draw = black]  ([xshift = 0cm,yshift = -0.2cm]p1.east) rectangle ([xshift = 0.4cm,yshift = 0.2cm]p1.east);
%%     \node(p2)[anchor =west ] at ([yshift = -0.5cm]p1.west) {$\pi(\text{long, loops}|S) = $};
%%     \draw [fill = yellow, fill opacity = 0.4,draw = black]  ([xshift = 0cm,yshift = -0.2cm]p2.east) rectangle ([xshift = 0.4cm,yshift = 0.2cm]p2.east);
%%     \node(p3)[anchor =west ] at ([yshift = -0.5cm]p2.west) {$\pi(\text{long, no loop/reflect}|S) = $};
%%     \draw [fill = yellow, fill opacity = 0.15,draw = black]  ([xshift = 0cm,yshift = -0.2cm]p3.east) rectangle ([xshift = 0.4cm,yshift = 0.2cm]p3.east);
%%     \node(p4)[anchor =west ] at ([yshift = -0.5cm]p3.west) {$\pi(\text{long, reflects}|S) = $};
%%     \draw [fill = yellow, fill opacity = 0.6,draw = black]  ([xshift = 0cm,yshift = -0.2cm]p4.east) rectangle ([xshift = 0.4cm,yshift = 0.2cm]p4.east);
%%     \node(p5)[anchor =west ] at ([yshift = -0.5cm]p4.west) {\emph{etc.}};
%%     \pause
%%     \node at (15.2,3) {\includegraphics[width = 3.6cm]{TikZfigures/2policyComparison}};
%%   \end{tikzpicture}}

%% \end{frame}

\begin{frame}{Top-down influences on perception}
  \begin{tabular}{ccc}
    \small{drawing\phantom{tttttttttttttttt}}&\small{bottom-up neural net}&\small{w/ top-down program bias}\\
    \multicolumn{3}{c}{\includegraphics[width = \textwidth]{TikZfigures/programSuccess16_statement.png}}\\
    \multicolumn{3}{c}{\only<3>{\includegraphics[width = \textwidth]{TikZfigures/programSuccess71_statement.png}}}
    %        \multicolumn{3}{c}{\includegraphics[width = \textwidth]{TikZfigures/programSuccess97.png}}
  \end{tabular}

  \only<2>{
    \begin{tabular}{cc}
      \begin{tabular}{c}
        \begin{tikzpicture}
        \node[obs](i) at (0,0) {img};
        \node[latent](p) at ([yshift=0.8cm]i.north) {prog};
        \node[latent](h) at ([yshift=0.8cm]p.north) {prior};
        \draw[->] (h.south) -- (p.north);
        \draw[->] (p.south) -- (i.north);
        \node[obs](i) at ([xshift=0.8cm]i.east) {img};
        \node[latent](p) at ([yshift=0.8cm]i.north) {prog};
        \draw[->] (h.south) -- (p.north);
        \draw[->] (p.south) -- (i.north);
        \node[obs](i) at ([xshift=-1.8cm]i.west) {img};
        \node[latent](p) at ([yshift=0.8cm]i.north) {prog};
        \draw[->] (p.south) -- (i.north);
        \draw[->] (h.south) -- (p.north);
      \end{tikzpicture}
        \end{tabular}&
      \begin{tabular}{l}
        $\text{predicted program} = $\\$\argmax_{\text{progs}}\probability\left[\text{img}|\text{prog} \right]\probability\left[\text{prog}|\text{prior} \right]$
        \end{tabular}
      \end{tabular}
  }
\end{frame}


\begin{frame}{3D program induction}
  \Wider[5em]{
    \begin{center}
      \includegraphics[width = 0.8\textwidth]{assets/we_are_cool_4.png}

      \vspace{0.52cm}
      
      \setlength{\tabcolsep}{0pt}
      \renewcommand{\arraystretch}{0}  
      
      %    \includegraphics[width = \textwidth]{assets/pixel_montage} \\
      % \begin{tabular}{cc}
      %   \includegraphics[width = 0.3\textwidth]{assets/2d_synthetic_montage}&
      %   \includegraphics[width = 0.6\textwidth]{assets/tool_montage}
      % \end{tabular}\\\\    
      \visible<2>{

        \begin{tabular}{ccccccc}
        \rotatebox[origin=l]{90}{Input}
        &    \rotatebox[origin=l]{90}{(voxels)}$\;$&
        \includegraphics[width = 2cm]{assets/3-D/demo3/spec}&
        %    \includegraphics[width = 2cm]{assets/3-D/demo6/0000png_v.png}&
        \includegraphics[width = 2cm]{assets/3-D/demo2/spec}&
        \includegraphics[width = 2cm]{assets/3-D/demo4/012_png_v.png}&
        \includegraphics[width = 2cm]{assets/3-D/demo5/CAD_example_input}&
        \includegraphics[width = 2cm]{assets/3-D/demo1/spec}
        \\
        \rotatebox[origin=l]{90}{Rendered}&
        \rotatebox[origin=l]{90}{program}$\;$&
        \includegraphics[width = 2cm]{assets/3-D/demo3/000_SMC_value_pickle_pretty.png}$\;$&
        %    \includegraphics[width = 2cm]{assets/3-D/demo6/2}$\;$&
        \includegraphics[width = 2cm]{assets/3-D/demo2/pretty}$\;$&
        \includegraphics[width = 2cm]{assets/3-D/demo4/012_SMC_value_pickle_pretty}$\;$&
        \includegraphics[width = 2cm]{assets/3-D/demo5/CAD_example_output}$\;$&
        \includegraphics[width = 2cm]{assets/3-D/demo1/pretty}
        \end{tabular}
        }
      \setlength{\tabcolsep}{6pt}
      \renewcommand{\arraystretch}{1}
    \end{center}
  }
  
  \vfill
\small  Ellis$^*$, Nye$^*$, Pu$^*$, Sosa$^*$, Tenenbaum, Solar-Lezama. NeurIPS 2019. $^*$equal contribution

\end{frame}


\begin{frame}{3D program induction}
  \Wider[5em]{
    \begin{center}
%      \includegraphics[width = 0.8\textwidth]{assets/we_are_cool_4.png}

 %     \vspace{0.52cm}
      
      \setlength{\tabcolsep}{0pt}
      \renewcommand{\arraystretch}{0}  
      
      %    \includegraphics[width = \textwidth]{assets/pixel_montage} \\
      % \begin{tabular}{cc}
      %   \includegraphics[width = 0.3\textwidth]{assets/2d_synthetic_montage}&
      %   \includegraphics[width = 0.6\textwidth]{assets/tool_montage}
      % \end{tabular}\\\\    


        \begin{tabular}{ccccccc}
        \rotatebox[origin=l]{90}{Spec}
        &    \rotatebox[origin=l]{90}{(voxels)}$\;$&
        \includegraphics[width = 2cm]{assets/3-D/demo3/spec}&
        %    \includegraphics[width = 2cm]{assets/3-D/demo6/0000png_v.png}&
        \includegraphics[width = 2cm]{assets/3-D/demo2/spec}&
        \includegraphics[width = 2cm]{assets/3-D/demo4/012_png_v.png}&
        \includegraphics[width = 2cm]{assets/3-D/demo5/CAD_example_input}&
        \includegraphics[width = 2cm]{assets/3-D/demo1/spec}
        \\
        \rotatebox[origin=l]{90}{Rendered}&
        \rotatebox[origin=l]{90}{program}$\;$&
        \includegraphics[width = 2cm]{assets/3-D/demo3/000_SMC_value_pickle_pretty.png}$\;$&
        %    \includegraphics[width = 2cm]{assets/3-D/demo6/2}$\;$&
        \includegraphics[width = 2cm]{assets/3-D/demo2/pretty}$\;$&
        \includegraphics[width = 2cm]{assets/3-D/demo4/012_SMC_value_pickle_pretty}$\;$&
        \includegraphics[width = 2cm]{assets/3-D/demo5/CAD_example_output}$\;$&
        \includegraphics[width = 2cm]{assets/3-D/demo1/pretty}
        \end{tabular}

      \setlength{\tabcolsep}{6pt}
      \renewcommand{\arraystretch}{1}

      \vspace{0.5cm}
      Challenge: combinatorics!\\
      Branching factor: $ > 1.3$ million per line of code, $\approx$ 20 lines of code

      
      
      \vspace{0.5cm}
      \visible<2>{
        Solution: stochastic \textbf{tree search} + learn \textbf{policy} that writes code\\ + learn \textbf{value} function that assesses execution of program so far;\\
      analogous to AlphaGo}
    \end{center}
  }
  
  \vfill
\small  Ellis$^*$, Nye$^*$, Pu$^*$, Sosa$^*$, Tenenbaum, Solar-Lezama. NeurIPS 2019. $^*$equal contribution

\end{frame}

\begin{frame}{Lessons}

  The bias from a programming language gives extrapolation, or strong generalization, sometimes even without meta-learning

  \vspace{1cm}
  
  Mix-and-match techniques to play to their strengths: neural nets for perception, symbols for reasoning, Bayesian methods for uncertainty

\end{frame}

\begin{frame}{}
  \begin{center}
    \begin{tabular}{l}
      {\textcolor{black}{Program Induction and }\textcolor{gray}{perception}}\\
      \phantom{Program Induction and }{\textcolor{black}{learning to learn}}\\
      \phantom{Program Induction and }{\textcolor{gray}{interpretable models}}
      \end{tabular}
  \end{center}
\end{frame}



\begin{frame}{Learning to write code} %Growing domain-specific knowledge}
  
  %  \Large
  Goal: acquire domain-specific knowledge needed to induce a class of programs


  
  \vspace{0.75cm}

  \Wider[4em]{
    \begin{itemize}
  \item Library of concepts (declarative knowledge; domain specific language) %; generative model over programs)
    \item Inference strategy (procedural knowledge; synthesis algorithm)
    \end{itemize}
  }
  
  \vspace{0.75cm}
  
  \only<2>{
  \begin{tikzpicture}
    \node(problem) at (0,0) {\includegraphics[width = 2cm]{figures/cubic.png}};
    \node(synthesizer)[draw,align=center] at ([xshift=3cm]problem.east) {Learned \\program inducer};
    \draw[->] (problem.east) -- (synthesizer.west);
    \node(program)[draw, align=center] at ([xshift=3cm]synthesizer.east) {program:\\$f(x) = 0.3x^3+$\\$1.1x^2-2x+0.6$};
    \draw[->] (synthesizer.east) -- (program.west);
  \end{tikzpicture}
  
    \vspace{0.2cm}Concepts: $x^3$, $\alpha x + \beta$, etc\\Inference strategy: neurosymbolic search for programs}
  \renewcommand\code\texttt
    \only<3>{
  \begin{tikzpicture}
    \node(problem) at (0,0) {\includegraphics[width = 2cm]{figures/radialCircle.png}};
    \node(synthesizer)[draw,align=center] at ([xshift=3cm]problem.east) {Learned \\program inducer};
    \draw[->] (problem.east) -- (synthesizer.west);
    \node(program)[draw, align=center] at ([xshift=3cm]synthesizer.east) {program:\\\code{(radial-symmetry 5}\\\code{ (circle 3))}};
    \draw[->] (synthesizer.east) -- (program.west);
  \end{tikzpicture}
  
  \vspace{0.2cm}Concepts: \code{circle}, \code{radial-symmetry}, etc\\Inference strategy: neurosymbolic search for programs
    }
\end{frame}

\begin{frame}{Library learning}
  \centering
  
  \only<1>{  \Wider[5em]{\includegraphics[width = \textwidth]{figures/figure15}}}
  \only<2>{  \Wider[5em]{\includegraphics[width = \textwidth]{figures/figure14}}}
  \only<3>{  \Wider[5em]{\includegraphics[width = \textwidth]{figures/figure13}}}
  \only<4>{  \Wider[5em]{\includegraphics[width = \textwidth]{figures/figure12}}}
  \only<5>{  \Wider[5em]{\includegraphics[width = \textwidth]{figures/figure11}}}
  
\end{frame}

\begin{frame}{DreamCoder}
  \begin{itemize}
  \item   \textbf{Wake:} Solve problems by writing programs
  \item \textbf{Sleep:} Improve library and neural recognition model:
    \begin{itemize}
    \item \textbf{Abstraction sleep:} Improve library
      \item \textbf{Dream sleep:} Improve neural inference model
    \end{itemize}
  \item   Combines ideas from Wake-Sleep \& Exploration-Compression % \& Program analysis
  \end{itemize}
\Wider[5.4em]{
  \begin{tikzpicture}
    \visible<2->{
      \node(Lisp) at (-3,0) {\includegraphics[width = 0.8\textwidth]{statement/taskbar2.png}};
    }
    \node at (2.5,1.3) {  \includegraphics[width = 4cm]{ecFigures/sleepingChild.jpg}};
    \end{tikzpicture}}


\end{frame}
\begin{frame}[t]{Library learning as Bayesian inference}
  %  \includegraphics[width = 11cm]{ecFigures/animation/EC.eps}
\centering  \begin{tikzpicture}[scale=1.3,line width=0.5mm]

  \node[latent,scale=1] at (3.5,3) (dx){Library};
  \node[latent,scale=1] at ([yshift=-1.5cm]dx) (zp){prog};
  \node[obs,scale=1] at ([yshift=-2cm]zp) (xp) {task};
  \node[latent,scale=1] at ([xshift=2cm]zp) (zp1){prog};
  \node[obs,scale=1] at ([xshift=2cm]xp) (xp1) {task};
  \draw [->] (zp1.south) -- (xp1.north);
  \draw [->] (dx.south) -- (zp1.north);
  %\draw [->,red] (xp1.east) to[out = 30,in = -30] node(nn){} (zp1.east);
%  \node at (nn) {\NeuralNetwork{0.5}};
  % HACK : "invisible" arrow to phantom and push to the left and align with next
  % slide
  \draw [->,red,opacity=0.0001] (xp1.east) to[out = 30,in = -30] node(nn){} (zp1.east);
  
  \node[latent,scale=1] at ([xshift=-2cm]zp) (zp1){prog};
  \node[obs,scale=1] at ([xshift=-2cm]xp) (xp1) {task};
  \draw [->] (zp1.south) -- (xp1.north);
  \draw [->] (dx.south) -- (zp1.north);
%  \draw [->,red] (xp1.east) to[out = 30,in = -30] node(nn){} (zp1.east);
%  \node at (nn) {\NeuralNetwork{0.5}};


%  \draw [->,red] (xp.east) to[out = 30,in = -30] node(nn){} (zp.east);
 % \node at (nn) {\NeuralNetwork{0.5}};
  \draw [->] (dx.south) -- (zp.north);
  \draw [->] (zp.south) -- (xp.north);


  %\node[shift={+(0,-1.7)}] at (nn) { $Q$  };

  \end{tikzpicture}
  

\vspace{0.5cm}
  
[Dechter et al., 2013]  [Liang et al, 2010]; [Lake et al, 2015]

%\textbf{Dechter et al.}: Exploration-Compression. Inspiration for DreamCoder.

  %% \vfill
  %% Gray: Observed.\\
  %% White: Latent.\\
  %% Boxed (plate): Repeated.\\
  
\end{frame}
\newcommand{\NeuralNetwork}[1]{    \begin{tikzpicture}[x=2.5cm,y=1.25cm,transform canvas={scale=#1,shift={+(-1,2.5)}}]
      \tikzstyle{neuron}=[circle,fill=blue!50,minimum size=20pt]
      \fill[fill=white] (-0.25,-0.5) rectangle (2.25,-4.5);
      \node[rectangle] at (1,1) {};
      \foreach \name / \y in {1,...,4}
          \node[neuron] (I-\name) at (0,-\y) {};
      \foreach \name / \y in {1,...,3}
          \node[neuron] (H-\name) at (1,-\y-0.5) {};
      \foreach \name / \y in {1,...,4}
          \node[neuron] (O-\name) at (2,-\y) {};
      \foreach \source in {1,...,4}
          \foreach \dest in {1,...,3}
              \draw [-latex] (I-\source) -- (H-\dest);
      \foreach \source in {1,...,3}
          \foreach \dest in {1,...,4}
              \draw [-latex] (H-\source) -- (O-\dest);
    \end{tikzpicture}}
\begin{frame}[t]{Library learning as \alert{neurally-guided} Bayesian inference}
\centering  \begin{tikzpicture}[scale=1.3,line width=0.5mm]

  \node[latent,scale=1] at (3.5,3) (dx){Library};
  \node[latent,scale=1] at ([yshift=-1.5cm]dx) (zp){prog};
  \node[obs,scale=1] at ([yshift=-2cm]zp) (xp) {task};
  \node[latent,scale=1] at ([xshift=2cm]zp) (zp1){prog};
  \node[obs,scale=1] at ([xshift=2cm]xp) (xp1) {task};
  \draw [->] (zp1.south) -- (xp1.north);
  \draw [->] (dx.south) -- (zp1.north);
  \draw [->,red] (xp1.east) to[out = 30,in = -30] node(nn){} (zp1.east);
%  \node at (nn) {\NeuralNetwork{0.5}};
  
  \node[latent,scale=1] at ([xshift=-2cm]zp) (zp1){prog};
  \node[obs,scale=1] at ([xshift=-2cm]xp) (xp1) {task};
  \draw [->] (zp1.south) -- (xp1.north);
  \draw [->] (dx.south) -- (zp1.north);
  \draw [->,red] (xp1.east) to[out = 30,in = -30] node(nn){} (zp1.east);
%  \node at (nn) {\NeuralNetwork{0.5}};


  \draw [->,red] (xp.east) to[out = 30,in = -30] node(nn){} (zp.east);
  \node at (nn) {\NeuralNetwork{0.25}};
  \draw [->] (dx.south) -- (zp.north);
  \draw [->] (zp.south) -- (xp.north);


  %\node[shift={+(0,-1.7)}] at (nn) { $Q$  };

\end{tikzpicture}


\Wider[4em]{
  \begin{center}
    library learning via program analysis + \\
    new neural inference network for program synthesis + \\
    better program representation (Lisp+polymorphic types [Milner 1978])
  \end{center}
}
\end{frame}
\newcommand{\NeuralNetwork}[1]{    \begin{tikzpicture}[x=2.5cm,y=1.25cm,transform canvas={scale=#1,shift={+(-1,2.5)}}]
      \tikzstyle{neuron}=[circle,fill=blue!50,minimum size=20pt]
      \fill[fill=teal!5!white] (-0.25,-0.5) rectangle (2.25,-4.5);
      \node[rectangle] at (1,1) {};
      \foreach \name / \y in {1,...,4}
          \node[neuron] (I-\name) at (0,-\y) {};
      \foreach \name / \y in {1,...,3}
          \node[neuron] (H-\name) at (1,-\y-0.5) {};
      \foreach \name / \y in {1,...,4}
          \node[neuron] (O-\name) at (2,-\y) {};
      \foreach \source in {1,...,4}
          \foreach \dest in {1,...,3}
              \draw [-latex] (I-\source) -- (H-\dest);
      \foreach \source in {1,...,3}
          \foreach \dest in {1,...,4}
              \draw [-latex] (H-\source) -- (O-\dest);
    \end{tikzpicture}}
\newcommand{\spiral}[2]{
  \draw[ultra thick,->] ([shift={#1}]-30:#2) arc [radius = #2, start angle = -30, end angle = 90];
  \draw[ultra thick,->] ([shift={#1}]-30:#2) arc [radius = #2, start angle = -30, end angle = 95];

  
      \draw[ultra thick,->] ([shift={#1}]90:#2) arc [radius = #2, start angle = 90, end angle = 210];
      \draw[ultra thick,->] ([shift={#1}]90:#2) arc [radius = #2, start angle = 90, end angle = 205];
      
      \draw[ultra thick,->] ([shift={#1}]210:#2) arc [radius = #2, start angle = 210, end angle = 340];
      \draw[ultra thick,->] ([shift={#1}]210:#2) arc [radius = #2, start angle = 210, end angle = 335];
}
\newcommand{\legend}{
  \begin{tikzpicture}
    \node at (0,0) (uses){is};
    \draw[->,red] ([xshift=-0.6cm]uses.west)  -- (uses.west);
    \node at ([xshift=0.4cm]uses.east) {\NeuralNetwork{0.15}};
    \draw[thin] (-1,-0.4) rectangle (1.2,0.4);
  \end{tikzpicture}
}

\begin{frame}{}
  \centering
  \only<1>{\includegraphics[width = 0.3\textwidth]{cycleGraphicalModel1}}
  \only<2>{\includegraphics[width = \textwidth]{cycleGraphicalModel2}}
  \only<3>{\includegraphics[width = \textwidth]{cycleGraphicalModel3}}
  \only<4>{\includegraphics[width = \textwidth]{cycleGraphicalModel4}}
  \only<5>{\includegraphics[width = \textwidth]{cycleGraphicalModel5}}
\end{frame}

%% \begin{frame}{DreamCoder's sleep cycles}
%% \tiny\begin{tikzpicture}[scale=0.55,line width=0.25mm]
%%     \draw[fill=teal!5!white] (-1.25,1.25) -- (13.25,1.25) -- (13.25,-4) -- (-1.25,-4) -- (-1.25,1.25);
%%     \node at (5.5,1.5) {\textsc{\textbf{Wake}}};

    
%%     \begin{scope}[shift={(0.5,0.5)}]
%%       \node[align=center] at (0,-0.5) (d){
%%         \begin{tabular}{l}
%%           \multicolumn{1}{c}{\textbf{Library}}\\
%%         \texttt{$f_1($x$)=$(+ x 1)}\\
%%         \texttt{$f_2($z$)=$(fold cons}\\
%%         \phantom{\texttt{$f_2($z$)$}}\texttt{(cons z nil))}\\
%%  $\cdots\cdots\cdots$
%%                 \end{tabular}};
%%       \node[align=center] at ([yshift = -2cm]d) (t){\textbf{Task}\\
%%                     \texttt{[7\, 2\, 3]}$\to$\texttt{[4 3 8]}         \\
%%             \texttt{[3\, 8]}$\to$\texttt{[9 4]}\\
%%             \texttt{[4\, 3\, 2]}$\to$\texttt{[3 4 5]} };

%%       \node at ([xshift = 1.25cm]t.east) (nn){\NeuralNetwork{0.1}};
%%       \node[align = center, text width = 1cm] at ([yshift = 0.7cm,xshift=0cm]nn.north) {\baselineskip=0pt  Recog. model\par};
%%       \draw [red,-{>[scale=0.2]}] (t.east) -- ([xshift = -0.5cm]nn.west);

%%       \node[draw,rounded corners, align=center, inner sep = 5] at ([xshift = 4.2cm,yshift = 1cm]t.east) (s){Neurally-Guided\\ Enumerative Search};

%%       \draw [red,->] ([xshift = 0.5cm]nn.east) -- ([yshift = -0.25cm]s.west);
%%       \draw [->,rounded corners,] (d.east) -- ([yshift = 2cm]nn.center) -- ([yshift = 0.25cm]s.west);

%%       \node[align=left] at ([xshift=3cm]s.east) (f) {\textbf{Programs for task:}\\
%%             \texttt{(map $f_1$ (fold $f_2$ nil x))}\\
%%          $\cdots\cdots\cdots$};
%%       \draw [->  ] (s.east) -- (f.west);

%%       \draw [->  ,rounded corners] (t.south) -- ([yshift = -0.5cm]t.south) -- ([yshift = -0.5cm] s.south |- t.south) -- (s.south);
%%     \end{scope}
    
%%     \begin{scope}[shift={(9.4,-3.5)},scale=0.6,line width=0.05mm]
%%       \node[obs,scale=0.7] at (3.5,3) (dx){Library};
%%       \node[latent,scale=0.7] at ([yshift=-1.7cm,xshift=0cm]dx) (zp){prog};
%%       \node[obs,scale=0.7] at ([yshift=-1.45cm]zp) (xp) {task};
%%       \node[latent,scale=0.7] at ([xshift=1.5cm]zp) (zp1){prog};
%%       \node[obs,scale=0.7] at ([xshift=1.5cm]xp) (xp1) {task};
%%       \draw [->] (zp1.south) -- (xp1.north);
%%       \draw [->] (dx.south) -- (zp1.north);
%%       \draw [->,red] (xp1.east) to[out = 30,in = -30] node(nn){} (zp1.east);
%%       \node[latent,scale=0.7] at ([xshift=-1.5cm]zp) (zp1){prog};
%%       \node[obs,scale=0.7] at ([xshift=-1.5cm]xp) (xp1) {task};
%%       \draw [->] (zp1.south) -- (xp1.north);
%%       \draw [->] (dx.south) -- (zp1.north);
%%       \draw [->] (dx.south) -- (zp.north);
%%       \draw [->] (zp.south) -- (xp.north);
%%       \draw [->,red] (xp1.east) to[out = 30,in = -30] node(nn){} (zp1.east);
%%       \draw [->,red] (xp.east) to[out = 30,in = -30] node(nn){} (zp.east);
%%     \end{scope}


%%     \node at (0,-4.75) {\textbf{\textsc{Sleep: Abstraction}}};
%%     \draw[fill=teal!5!white] (-3,-5) -- (3,-5) -- (3,-10) -- (5.5,-10) -- (5.5,-13) -- (-3,-13) -- (-3,-5);
%%     \node at (12,-4.75) {\textbf{\textsc{Sleep: Dreaming}}};
%%     \draw[fill=teal!5!white] (15,-5) -- (9,-5) -- (9,-10) -- (6.5,-10) -- (6.5,-13) -- (15,-13) -- (15,-5);

%%     \begin{scope}[shift={(9.5,-4.5)}]
%%       \node(dreaming) at (1,-1) {\underline{Fantasies}};
%%       \node[anchor=center] at ([yshift=-0.5cm]dreaming.south) (d){\textbf{Library}};
%%       \node at ([yshift=-1.75cm]d.south) (p1){program};
%%       \draw[squiggle,-> ] (d.south) -- node[sloped,above]{{ sample}} (p1.north);

%%       \node(replay) at ([xshift=2cm]dreaming.east) {\underline{Replays}};
%%       \node[anchor=center,align=center] at ([yshift=-0.5cm]replay.south) (d){\textbf{progs. for task}};
%%       \node at ([yshift=-1.75cm]d.south) (p1){program};
%%       \draw[squiggle,-> ] (d.south) -- node[sloped,above]{ sample} (p1.north);

%%       \node(p1) at (1.5,-6) {program};      
%%       \node at ([xshift = 2.0cm]p1.east) (t1){ task};
%%       \draw [-> ] (p1.east) -- node[above]{ run} (t1.west);
%%       \node(n) at ([yshift=-1.2cm,xshift=1.25cm]p1.south) {
%%         \NeuralNetwork{0.17}};
%%       \draw [->,red] (t1.south) to[out = -90,in = 0]  ([xshift=0.4cm]n.east);
%%       \draw [dashed] (p1.south) to[out=-120,in=180] node[above,fill=teal!5!white]{\color{black}Loss} ([xshift=-0.4cm]n.west);


%%       \node at ($(-0.25,0.5) + (p1.north)!0.5!(t1.north)$) {\underline{Train recognition model}};

%%       %% \node at ([xshift = 1.5cm]p1.east) (t1){ task};
%%       %% \draw [-> ] (p1.east) -- node[above]{ run} (t1.west);
%%       %% \node(n) at ([yshift=-1.2cm,xshift=0.4cm]p1.south) {
%%       %%   \NeuralNetwork{0.17}};
%%       %% \draw [->,red] (t1.south) to[out = -90,in = 0]  ([xshift=0.4cm]n.east);
%%       %% \draw [dashed] (p1.south) to[out=-120,in=180] node[above,fill=white]{\color{black}Loss} ([xshift=-0.4cm]n.west);

%%       \begin{scope}[shift={(-3.8,-8)},scale=0.6,line width=0.05mm]
%%         \node[obs,scale=0.4] at (3.5,3) (dx){L};
%%         \node[obs,scale=0.4] at ([yshift=-1.7cm,xshift=0cm]dx) (zp){p};
%%         \node[obs,scale=0.4] at ([yshift=-1.45cm]zp) (xp) {t};
%%         \node[obs,scale=0.4] at ([xshift=1.5cm]zp) (zp1){p};
%%         \node[obs,scale=0.4] at ([xshift=1.5cm]xp) (xp1) {t};
%%         \draw [->] (zp1.south) -- (xp1.north);
%%         \draw [->] (dx.south) -- (zp1.north);
%%         \draw [->,red] (xp1.east) to[out = 30,in = -30] node(nn){} (zp1.east);
%%         \node[obs,scale=0.4] at ([xshift=-1.5cm]zp) (zp1){p};
%%         \node[obs,scale=0.4] at ([xshift=-1.5cm]xp) (xp1) {t};
%%         \draw [->] (zp1.south) -- (xp1.north);
%%         \draw [->] (dx.south) -- (zp1.north);
%%         \draw [->] (dx.south) -- (zp.north);
%%         \draw [->] (zp.south) -- (xp.north);
%%         \draw [->,red] (xp1.east) to[out = 30,in = -30] node(nn){} (zp1.east);
%%         \draw [->,red] (xp.east) to[out = 30,in = -30] node(nn){} (zp.east);
%%       \end{scope}


%%       \end{scope}

%%     % memory consolidation
%%     \begin{scope}[shift={(-2,-4.5)}]

%%       %% defined routines for creating fragmented syntax trees
%%       \newcommand{\syntaxOne}[1]{
%%         \begin{tikzpicture}[scale=#1,line width=0.35mm]          
%%           \node(l1) at (0,0) {};
%%           \node[color=pop3](p1) at (-1,-1) {\texttt{+}};
%%           \node[color=pop3](n1) at (0.7,-0.9) {\texttt{1}};
%%           \node(x1) at (0,-1) {\texttt{1}};
%%           \draw[color=pop3] (l1.south) -- (p1.north);
%%           \draw[color=pop3] (l1.south) -- (n1.north);
%%           \draw[color=pop3] (-0.5,-0.45) -- (x1.north);

%%           \node(t) at (-0.5,0.5) {};
%%           \draw (l1.south) -- (t.south);
%%           \node(c) at (-1.5,-0.2) {\texttt{cons}};
%%           \draw (t.south) -- (c.north);
%%         \end{tikzpicture}
%%       }
%%       \newcommand{\syntaxTo}[1]{
%%         \begin{tikzpicture}[scale=#1,line width=0.35mm]          
%%             \node(l1) at (0,0) {};
%%             \node[color=pop3](p1) at (-1,-1) {\texttt{+}};
%%             \node[color=pop3](n1) at (0.7,-0.9) {\texttt{1}};
%%             \draw[color=pop3] (l1.south) -- (p1.north);
%%             \draw[color=pop3] (l1.south) -- (n1.north);
%%             \draw[color=pop3] (-0.5,-0.45) -- (0,-1);
%%             \node(c) at (-0.5,-1.5) {\texttt{car}};
%%             \node(z) at (0.5,-1.5) {\texttt{z}};
%%             \draw (0,-1) -- (c.north);
%%             \draw (0,-1) -- (z.north);
%%         \end{tikzpicture}
%%       }

%%       \node[align=center,anchor=center] at (0.4,-1.2) (f1){\textbf{ progs. for task 1}:\\\texttt{(+ (car z) 1)}};
%%       \node[align=center] at ([xshift = 1.75cm]f1.east) (f2){\textbf{  progs. for task 2}:\\\texttt{(cons (+ 1 1))}};
%%       \node(s1) at ([yshift=-0.5cm]f1.south) {\syntaxOne{0.5}};
%%       \node(s2) at ([yshift=-0.5cm]f2.south) {\syntaxTo{0.5}};
%%       \node(c)[align=center,rectangle, rounded corners, draw, minimum width = 1cm, minimum height = 0.5cm, anchor = north] at ($(s1.south)!0.5!(s2.south) + (0,-1)$) {Refactoring Algorithm};
%%       \draw [-> ] (s1.south) -- (s1.south|-c.north);
%%       \draw [-> ] (s2.south) -- (s2.south|-c.north);

      
%%       \node(d) at ([yshift = -1.8cm]c.south) {
%%         \begin{tikzpicture}[scale=0.5,line width=0.5mm]
%%           \node[align=center] at (0,0) {\textbf{new Library} w/ \texttt{(+ x 1)}:};
%%           \begin{scope}[shift={(0.6,-0.5)}]
%%             \node[pop3](p1) at (-1,-1) {\texttt{+}};
%%             \node[pop3](n1) at (0.6,-0.9) {\texttt{1}};
%%             \node[pop3](a) at (0,-1) {\texttt{ }};
%%             \draw[pop3] (0,0) -- (p1.north);
%%             \draw[pop3] (0,0) -- (n1.north);
%%             \draw[pop3] (-0.3,-0.3) -- (a.north);
%%           \end{scope}
%%       \end{tikzpicture}};
%%       \draw [-> ] (c.south) -- (d.north);



%%       \begin{scope}[shift={(4,-8)},scale=0.6,line width=0.05mm]
%%         \node[latent,scale=0.7] at (3.5,3) (dx){Library};
%%         \node[obs,scale=0.7] at ([yshift=-1.7cm,xshift=0cm]dx) (zp){prog};
%%         \node[obs,scale=0.7] at ([yshift=-1.45cm]zp) (xp) {task};
%%         \node[obs,scale=0.7] at ([xshift=1.5cm]zp) (zp1){prog};
%%         \node[obs,scale=0.7] at ([xshift=1.5cm]xp) (xp1) {task};
%%         \draw [->] (zp1.south) -- (xp1.north);
%%         \draw [->] (dx.south) -- (zp1.north);
%%         \node[obs,scale=0.7] at ([xshift=-1.5cm]zp) (zp1){prog};
%%         \node[obs,scale=0.7] at ([xshift=-1.5cm]xp) (xp1) {task};
%%         \draw [->] (zp1.south) -- (xp1.north);
%%         \draw [->] (dx.south) -- (zp1.north);
%%         \draw [->] (dx.south) -- (zp.north);
%%         \draw [->] (zp.south) -- (xp.north);
%%       \end{scope}


%%       \end{scope}


    
%%     %% center spiral
%%     \begin{scope}[shift={(3.25,-7.9)},scale=0.8]    
%%       \spiral{(3.5,1)}{3.5}
%%       \node[latent,scale=1] at (3.5,3) (dx){Library};
%%       \node[latent,scale=1] at ([yshift=-2cm,xshift=0cm]dx) (zp){prog};
%%       \node[obs,scale=1] at ([yshift=-1.45cm]zp) (xp) {task};
%%       \node[latent,scale=1] at ([xshift=2cm]zp) (zp1){prog};
%%       \node[obs,scale=1] at ([xshift=2cm]xp) (xp1) {task};
%%       \draw [->] (zp1.south) -- (xp1.north);
%%       \draw [->] (dx.south) -- (zp1.north);
%%       \draw [->,red] (xp1.east) to[out = 30,in = -30] node(nn){} (zp1.east);
      
%%       \node[latent,scale=1] at ([xshift=-2cm]zp) (zp1){prog};
%%       \node[obs,scale=1] at ([xshift=-2cm]xp) (xp1) {task};
%%       \draw [->] (zp1.south) -- (xp1.north);
%%       \draw [->] (dx.south) -- (zp1.north);
%%       \draw [->,red] (xp1.east) to[out = 30,in = -30] node(nn){} (zp1.east);


%%       \draw [->,red] (xp.east) to[out = 30,in = -30] node(nn){} (zp.east);
%%       \draw [->] (dx.south) -- (zp.north);
%%       \draw [->] (zp.south) -- (xp.north);

%%       \node at ([yshift=-0.6cm]xp.south) {\legend};

%%     \end{scope}    
%%   \end{tikzpicture}  

%% \end{frame}

%% \begin{frame}{}
%%   \only<1>{\includegraphics[width = \textwidth]{figures/compressionAnimation/1.jpg}}
%%   \only<2>{\includegraphics[width = \textwidth]{figures/compressionAnimation/2.jpg}}
%%   \only<3>{\includegraphics[width = \textwidth]{figures/compressionAnimation/3.jpg}}
%%   \only<4>{\includegraphics[width = \textwidth]{figures/compressionAnimation/4.jpg}}
%%   \only<5>{\includegraphics[width = \textwidth]{figures/compressionAnimation/5.jpg}}
%% %  \only<6>{\includegraphics[width = \textwidth]{figures/compressionAnimation/6.jpg}}
%%   \only<7>{\includegraphics[width = \textwidth]{figures/compressionAnimation/7.jpg}}
%%   \only<8>{\includegraphics[width = \textwidth]{figures/compressionAnimation/8.jpg}}
%%   \only<9>{\includegraphics[width = \textwidth]{figures/compressionAnimation/9.jpg}}
%%   \only<10>{\includegraphics[width = \textwidth]{figures/compressionAnimation/10.jpg}}
%%   \only<11>{\includegraphics[width = \textwidth]{figures/compressionAnimation/11.jpg}}
%%   \only<12>{\includegraphics[width = \textwidth]{figures/compressionAnimation/12.jpg}}
%%   \only<13>{\includegraphics[width = \textwidth]{figures/compressionAnimation/13.jpg}}
%%   \only<14>{\includegraphics[width = \textwidth]{figures/compressionAnimation/14.jpg}}
%%   \only<15>{\includegraphics[width = \textwidth]{figures/compressionAnimation/15.jpg}}
%%   \only<16>{\includegraphics[width = \textwidth]{figures/compressionAnimation/16.jpg}}
%%   \only<17>{\includegraphics[width = \textwidth]{figures/compressionAnimation/17.jpg}}
%%   \only<18>{\includegraphics[width = \textwidth]{figures/compressionAnimation/18.jpg}}
%%   \only<19>{\includegraphics[width = \textwidth]{figures/compressionAnimation/19.jpg}}
%%   \only<20>{\includegraphics[width = \textwidth]{figures/compressionAnimation/20.jpg}}
%%   \only<21>{\includegraphics[width = \textwidth]{figures/compressionAnimation/21.jpg}}
%%   \only<22>{\includegraphics[width = \textwidth]{figures/compressionAnimation/22.jpg}}
%%   \only<23>{\includegraphics[width = \textwidth]{figures/compressionAnimation/23.jpg}}
%%   \only<24>{\includegraphics[width = \textwidth]{figures/compressionAnimation/24.jpg}}
%%   \only<25>{\includegraphics[width = \textwidth]{figures/compressionAnimation/25.jpg}}
%%   \only<26>{\includegraphics[width = \textwidth]{figures/compressionAnimation/26.jpg}}
%% \end{frame}
\newcommand{\abstractionTree}[2]{
  \smallTree{#1}{#2}{+}{$x$}{$x$};
  \node(a#1) at ([yshift=0.5cm]root#1.south) {$\lambda x$};

  \draw (a#1.south) -- (root#1.south);  
  }
\newcommand{\smallTree}[5]{
  \node(root#1) at (#2) {};
  \node[anchor=north](r#1) at ([xshift=0.5cm,yshift=-0.5cm]root#1) {#5};
  \node(l#1) at ([xshift=-0.5cm,yshift=-0.5cm]root#1) {};
  \node(ll#1) at ([xshift=-0.5cm,yshift=-0.5cm]l#1) {#3};
  \node(lr#1) at ([xshift=0.5cm,yshift=-0.5cm]l#1) {#4};

  \draw (root#1.south) -- (r#1.north);
  \draw (root#1.south) -- (l#1.north);
  \draw (l#1.north) -- (ll#1.north);
  \draw (l#1.north) -- (lr#1.north);
}
\newcommand{\appliedTree}[6]{
  \smallTree{#1}{#2}{#3}{#4}{#5};
  \node(a#1) at ([yshift=0.5cm]root#1.south) {$\lambda x$};
  \node(ar#1) at ([xshift=0.5cm,yshift=0.4cm]a#1.north) {};
  \node[anchor=north](argument#1) at ([xshift=1cm]a#1.north) {#6};

  \draw (a#1.south) -- (root#1.south);
  \draw (argument#1.north) -- (ar#1.south);
  \draw (a#1.north) -- (ar#1.south);
}
\newcommand{\chooseAny}[4]{
  \node[anchor=north](any#1) at (#2) {\textsc{\small Any}};
  \node(anyl#1) at ([xshift=-0.25cm,yshift=-0.5cm]any#1.south) {#3};
  \node(anyr#1) at ([xshift=0.25cm,yshift=-0.5cm]any#1.south) {#4};
  \draw (any#1.south)--(anyl#1.north);
  \draw (any#1.south)--(anyr#1.north);  
}
\newcommand{\chooseTreeTiny}[3]{
  \node(a#1) at (#2) {$\lambda z$};
  \node(ar#1) at ([xshift=0.5cm,yshift=0.4cm]a#1.north) {};
  \node[anchor=north](argument#1) at ([xshift=1cm]a#1.north) {#3};

  \chooseAny{any#1}{[yshift=-0.2cm]a#1.south}{#3}{$z$}

  \draw (a#1.south) -- (anyany#1.north);
  \draw (argument#1.north) -- (ar#1.south);
  \draw (a#1.north) -- (ar#1.south);
}  

\newcommand{\chooseTreeSmall}[7]{
  \node(root#1) at (#2) {};
  \node[anchor=north](r#1) at ([xshift=0.5cm,yshift=-0.5cm]root#1) {#5};
  \node(l#1) at ([xshift=-0.5cm,yshift=-0.5cm]root#1) {};
  \chooseAny{any#1}{[xshift=-0.5cm,yshift=-0.5cm]l#1}{#3}{#4};
  \draw (l#1.north) -- (anyany#1.north);
%  \node(ll#1) at ([xshift=-0.5cm,yshift=-0.5cm]l#1) {};
  \node(lr#1) at ([xshift=0.5cm,yshift=-0.5cm]l#1) {#5};

  \draw (root#1.south) -- (r#1.north);
  \draw (root#1.south) -- (l#1.north);
  \draw (l#1.north) -- (lr#1.north);

  \node(a#1) at ([yshift=0.5cm]root#1.south) {$\lambda y$};
  \node(ar#1) at ([xshift=0.5cm,yshift=0.4cm]a#1.north) {};
  \node[anchor=north](argument#1) at ([xshift=1cm]a#1.north) {#7};

  \draw (a#1.south) -- (root#1.south);
  \draw (argument#1.north) -- (ar#1.south);
  \draw (a#1.north) -- (ar#1.south);
}
\newcommand{\chooseTree}[6]{
  \node(root#1) at (#2) {};
  \chooseAny{anyl#1}{[xshift=1cm,yshift=-0.8cm]root#1}{#4}{#5};
  \draw (root#1.south) -- (anyanyl#1.north);  
  \node(l#1) at ([xshift=-0.75cm,yshift=-0.75cm]root#1) {};
  \node(ll#1) at ([xshift=-0.75cm,yshift=-0.75cm]l#1) {#3};
  \chooseAny{anyr#1}{[xshift=0.5cm,yshift=-0.5cm]l#1}{#4}{#5};
  \draw (l#1.north) -- (anyanyr#1.north);

  \draw (root#1.south) -- (l#1.north);
  \draw (l#1.north) -- (ll#1.north);


  \node(a#1) at ([yshift=0.5cm]root#1.south) {$\lambda x$};
  \node(ar#1) at ([xshift=0.5cm,yshift=0.4cm]a#1.north) {};
  \node[anchor=north](argument#1) at ([xshift=1cm]a#1.north) {#6};

  \draw (a#1.south) -- (root#1.south);
  \draw (argument#1.north) -- (ar#1.south);
  \draw (a#1.north) -- (ar#1.south);
}
\newcommand{\chooseTreeCompact}[6]{
  \node(root#1) at (#2) {};
  \chooseAny{anyl#1}{[xshift=0.75cm,yshift=-0.5cm]root#1}{#4}{#5};
  \draw (root#1.south) -- (anyanyl#1.north);  
  \node(l#1) at ([xshift=-0.5cm,yshift=-0.5cm]root#1) {};
  \node(ll#1) at ([xshift=-0.5cm,yshift=-0.5cm]l#1) {#3};
  \chooseAny{anyr#1}{[xshift=0.3cm,yshift=-0.3cm]l#1}{#4}{#5};
  \draw (l#1.north) -- (anyanyr#1.north);

  \draw (root#1.south) -- (l#1.north);
  \draw (l#1.north) -- (ll#1.north);


  \node(a#1) at ([yshift=0.5cm]root#1.south) {$\lambda x$};
  \node(ar#1) at ([xshift=0.5cm,yshift=0.4cm]a#1.north) {};
  \node[anchor=north](argument#1) at ([xshift=1cm]a#1.north) {#6};

  \draw (a#1.south) -- (root#1.south);
  \draw (argument#1.north) -- (ar#1.south);
  \draw (a#1.north) -- (ar#1.south);
}
  
  
\begin{frame}[fragile]{Abstraction Sleep: Growing the library via refactoring}
  $5 + 5$\\\pause\texttt{(+ 5 5)}\pause

  \begin{center}
    \begin{tikzpicture}[scale=2]
      \smallTree{start}{0,0}{+}{5}{5};
    \end{tikzpicture}
  \end{center}
\end{frame}

\begin{frame}[fragile]{Abstraction Sleep: Growing the library via refactoring}
  \centering
  \begin{tikzpicture}[scale=1]
    \smallTree{start}{0,0}{+}{5}{5};
    \pause
    \appliedTree{r1}{4,-0.5}{+}{5}{$x$}{5};
    \pause
    \appliedTree{r2}{0,2}{$+$}{$x$}{5}{5};
    \pause
    \appliedTree{r3}{-4,-0.5}{$+$}{$x$}{$x$}{5};
    \pause
    \appliedTree{r4}{0,-3}{$+$}{5}{5}{5};
    \pause

    \draw [*-*,dashed,orange, very thick](rootstart.south) to[out=10,in=150] (arr1.south);
    \draw [-*,dashed,orange, very thick](rootstart.south) to[out=20,in=-90] ([xshift=2cm,yshift=-1cm]arr2.south) to[out=90,in=10] (arr2.south);
    \draw [-*,dashed,orange, very thick](rootstart.south) to[out=160,in=30] (arr3.south);
    \draw [-*,dashed,orange, very thick](rootstart.south) to[out=20,in=0] (arr4.south);

    \node[anchor=south west](legend) at (-5,3) {legend};
    \draw [dashed,orange, very thick]([xshift=0.1cm,yshift=-0.2cm]legend.south west)--([yshift=-0.2cm,xshift=0.5cm]legend.south west);
    \node(se)[anchor=west] at ([yshift=-0.2cm,xshift=0.5cm]legend.south west) {\small semantic equivalence};
    \draw[rounded corners] (legend.north west) rectangle (se.south east);

    \node[anchor=west] at (-5.5,-4) {\small cf. [Tate et al. 2009]};
    
  \end{tikzpicture}
\end{frame}

\begin{frame}[fragile]{Abstraction Sleep: Growing the library via refactoring}
\Wider[4em]{  \begin{tikzpicture}
    \node[anchor=south west](legend) at (-5,2) {legend};
    \draw [dashed,orange, very thick]([xshift=0.1cm,yshift=-0.2cm]legend.south west)--([yshift=-0.2cm,xshift=0.9cm]legend.south west);
    \node(se)[anchor=west] at ([yshift=-0.2cm,xshift=0.9cm]legend.south west) {\small semantic equivalence};
    \node[anchor=west](choosing) at ([yshift=-0.6cm]legend.south west) {{\textsc{\small Any}} \small nondeterministic choice};
    \draw[rounded corners] (legend.north west) rectangle (choosing.south east);

    \smallTree{start}{0,0}{+}{5}{5};
    \chooseTree{L}{4,-1}{+}{5}{$x$}{5};
    \draw [*-*,dashed,orange, very thick] (rootstart.south)    -- (arL.south);

    \node[align=left,anchor=west] at (-5.75,-5) {cf. Tate et al. 2009\\\phantom{cf. }Gulwani 2012};

    \pause

    \chooseTreeSmall{R}{-4,-1}{+}{$y$}{5}{5}{+};
    \draw [-*,dashed,orange, very thick] (rootstart.south)    -- (arR.south);

    \pause

    \chooseTreeTiny{fiber}{0,-3}{5};
    
    \draw [-*,dashed,orange, very thick] (rstart.south)    -- (arfiber.south);
    \draw [dashed,orange, very thick] (lrstart.south)    -- (arfiber.south);
    \draw [dashed,orange, very thick] (rR.east)    -- (arfiber.south);
    \draw [dashed,orange, very thick] (lrR.south east)    -- (arfiber.south);

    \pause

    \node[inner sep=0pt,dotted,very thick,label=below:{2 choices},fit=(anyanyfiber)(anylanyfiber)(anyranyfiber),draw] {};
    \node[inner sep=0pt,dotted,very thick,label=below:{2 choices},fit=(anyanyR)(anylanyR)(anyranyR),draw] {};
    \node[inner sep=0pt,dotted,very thick,label=below:{2 choices},fit=(anyanylL)(anyranylL)(anylanylL),draw] {};
    \node[inner sep=0pt,dotted,very thick,label=below:{2 choices},fit=(anyanyrL)(anyranyrL)(anylanyrL),draw] {};

    \pause
    \node at (arL.north) {\small $2\times 2=4$ choices};
    \node at (arR.north) {\small $2\times 2\times 2=8$ choices};
    \node at (rootstart.north) {\small $2\times 2=4$ choices};
%    \node[inner sep=0pt,dotted,very thick,label=above:{$2\times 2=4$ choices},fit=(choiceLabelA)(choiceLabelB),draw] {};
    
  \end{tikzpicture}}
\end{frame}

\begin{frame}[fragile]{}
\Wider[4em]{  \begin{tikzpicture}
    \node[anchor=south west](legend) at (-1,-8) {legend};
    \draw [dashed,orange, very thick]([xshift=0.1cm,yshift=-0.2cm]legend.south west)--([yshift=-0.2cm,xshift=0.9cm]legend.south west);
    \node(se)[anchor=west] at ([yshift=-0.2cm,xshift=0.9cm]legend.south west) {\small semantic equivalence};
    \node[anchor=west](choosing) at ([yshift=-0.6cm]legend.south west) {{\textsc{\small Any}} \small nondeterministic choice};
    \draw[rounded corners] (legend.north west) rectangle (choosing.south east);
    
    \smallTree{original}{0,0}{+}{5}{5};
    \chooseTreeCompact{originalSpace}{3,-1}{+}{5}{$x$}{5};
    \draw [*-*,dashed,orange, very thick] (rootoriginal.south)    -- (aroriginalSpace.south);

    \smallTree{remix}{0,-4}{+}{4}{4};
    \chooseTreeCompact{remixSpace}{3,-5}{+}{4}{$x$}{4};
    \draw [*-*,dashed,orange, very thick] (rootremix.south)    -- (arremixSpace.south);

    \pause

    \draw[very thick] ([yshift=-1cm]rootremixSpace.north) circle (1.4cm);
    \draw[very thick] ([yshift=-1cm]rootoriginalSpace.north) circle (1.4cm);
    \node at ([xshift=1.5cm,yshift=-0.2cm]rootoriginalSpace.north) {\textbf{``A''}};
    \node at ([xshift=1.5cm,yshift=-0.2cm]rootremixSpace.north) {\textbf{``B''}};

    \pause

    \abstractionTree{common}{8,-1};
    \node[anchor=south] at (acommon.north) {\textbf{A}$\mathbf\cap$\textbf{B is:}};

    \pause

    \node(python) at (lrcommon.south) {$\underbrace{\phantom{testingtestinging}}_{\text{\texttt{(lambda (x) (+ x x))}}}$};

    \pause

    \node(double)[anchor=north] at (python.south) {$=$\texttt{double}};

    \pause

    \node[anchor=north] at ([yshift=-1cm]double.south) {
        \begin{tabular}{ll}
          w/o \texttt{double}&w/ \texttt{double}\\\midrule 
          \texttt{(+ 5 5)}&\texttt{(double 5)}\\
          \texttt{(+ 4 4)}&\texttt{(double 4)}\\
          \texttt{(+ 3 3)}&\texttt{(double 3)}\\
          \multicolumn{2}{c}{$\cdots $}
      \end{tabular}};

    

    \end{tikzpicture}}

  \end{frame}

\begin{frame}{Abstraction Sleep: Growing the library via refactoring}
  \centering
    \Wider[0em]{\begin{tikzpicture}[every node/.style={inner sep=1,outer sep=0,rounded corners,thick, scale=0.7}]
  \footnotesize
  \node(p1)[draw,rounded corners,thick] at (-1,0) {
    \begin{tabular}{l}
      \texttt{(Y ($\lambda$ (r l) (if (nil? l) nil}\\
      \texttt{ (cons (+ (car l) (car l))}\\
      \phantom{\texttt{(cons }}\texttt{ (r (cdr l))))))}
    \end{tabular}
  };
  
  \node(p2)[draw] at ([xshift=4cm]p1.east) {
    \begin{tabular}{l}
      \texttt{(Y ($\lambda$ (r l) (if (nil? l) nil}\\
      \texttt{ (cons (- (car l) 1)}\\
      \phantom{\texttt{(cons }}\texttt{ (r (cdr l))))))}
    \end{tabular}
    
  };

    \node(t1)[draw] at ([yshift=1cm]p1.north) {\begin{tabular}{ll}
      \textbf{Task}:&\texttt{(1 2 3)$\to$(2 4 6)}\\
      &\texttt{(4 3 4)$\to$(8 6 8)}
  \end{tabular}};
  \draw [->] (t1.south)  --(p1.north) node[fill=white,midway] {Wake: program search};
  \node(t2)[draw] at ([yshift=1cm]p2.north) {\begin{tabular}{ll}
      \textbf{Task}:&\texttt{(1 2 3)$\to$(0 1 2)}\\
      &\texttt{(4 3 4)$\to$(3 2 3)}
  \end{tabular}};
  \draw [->] (t2.south)  --(p2.north) node[fill=white,midway] {Wake: program search};

  
  \pause
  \node(r1)[draw,inner sep=0,outer sep=0] at ([yshift=-2cm]p1.south) {
    \begin{tabular}{l}
      \texttt{(}\orange{\texttt{($\lambda$ (f) (Y ($\lambda$ (r l) (if (nil? l)}}\\
      \phantom{(($\lambda$ (f) (Y ($\lambda$ (r l)}\orange{\texttt{nil}}\\
      \phantom{(($\lambda$ (f) (Y ($\lambda$ (r l)}\orange{\texttt{(cons (f (car l))}}\\
      \phantom{(($\lambda$ (f) (Y ($\lambda$ (r l)}\orange{\texttt{ (r (cdr l)))))))}}\\
      \texttt{ ($\lambda$ (z) (+ z z)))}
    \end{tabular}
  };

  \node(r2)[draw] at ([yshift=-2cm]p2.south) {
    \begin{tabular}{l}
      \texttt{(}\orange{\texttt{($\lambda$ (f) (Y ($\lambda$ (r l) (if (nil? l)}}\\
      \phantom{(($\lambda$ (f) (Y ($\lambda$ (r l)}\orange{\texttt{nil}}\\
      \phantom{(($\lambda$ (f) (Y ($\lambda$ (r l)}\orange{\texttt{(cons (f (car l))}}\\
      \phantom{(($\lambda$ (f) (Y ($\lambda$ (r l)}\orange{\texttt{ (r (cdr l)))))))}}\\
      \texttt{ ($\lambda$ (z) (- z 1)))}
    \end{tabular}

  };

  \draw [->] (p1.south)  --(r1.north) node[fill=white,midway,align=center] {refactor\\($10^{14}$ refactorings)};
  \draw [->] (p2.south)  --(r2.north) node[fill=white,midway,align=center] {refactor\\($10^{14}$ refactorings)};



    \node(dummy) at ($(0,1) + (r1.north)!0.5!(r2.north)$) {};
    \node(dummy1) at (r1.west) {\phantom{t}};


    \draw[ultra thick] ($(r1.north west) + (-0.5,1)$) -- ($(r2.north east) + (0.5,1)$)
    -- ($(r2.north east) + (0.5,1) + (0,-5.75)$)
    -- ($(r1.north west) + (-0.5,1) + (0,-5.75)$)
    -- ($(r1.north west) + (-0.5,1)$);
    %  \node(sleepBox)[ultra thick, rounded corners=0, inner sep=25,outer sep=20, draw, fit= (dummy) (r1) (r2) (m) (dummy1) ] {};
    \node at ($(0.0,0.5) + (r1.north)!0.5!(r2.north)$) {{\normalsize\textbf{Sleep: Abstraction}}};

    \pause
    
    \node[draw](m) at ($(0,-2) + (r1.south)!0.5!(r2.south)$) {
      \begin{tabular}{lr}
        &\\
        \code{(}\fbox{\textsc{map}}\code{ ($\lambda$ (z) (+ z z)))}&
        \code{(}\fbox{\textsc{map}}\code{ ($\lambda$ (z) (- z 1)))}\\&\\
        \multicolumn{2}{l}{\fbox{\textsc{map}} = \orange{\texttt{($\lambda$ (f) (Y ($\lambda$ (r l) (if (nil? l) nil}}}\\
        \multicolumn{2}{l}{\phantom{\texttt{\emph{map}} = \texttt{($\lambda$ (f) (Y ($\lambda$ (r l) (if }}\orange{\texttt{(cons (f (car l))}}}\\
        \multicolumn{2}{l}{\phantom{\texttt{\emph{map}} = \texttt{($\lambda$ (f) (Y ($\lambda$ (r l) (if }}\orange{\texttt{(r (cdr l))))))}}}
      \end{tabular}      
    };
    \draw [->](r1.south)--($(-1.75,-0.3) + (m.north)$);
    \draw [->](r2.south)--($(1.75,-0.3) + (m.north)$);
    \node[fill=white] at ([yshift=0.4cm]m.north) {\textbf{Compress (MDL/Bayes objective)}};

    \end{tikzpicture}}
\end{frame}

\begin{comment}
\begin{frame}{Version space algebra for refactoring}
  \begin{tabular}{lr}
    \begin{tabular}{l}
      Expr\phantom{$\vert$}$\to $ var\\
      \phantom{Exprtt}$\vert$ $\lambda $var.Expr\\
      \phantom{Exprtt}$\vert$ (Expr Expr)\\
      \phantom{Exprtt}$\vert$ primitive\\
      \phantom{VStt$\vert$ VS$\uplus$VS}    \\
      \phantom{VStt$\vert$ $\Lambda$}    \\
      \phantom{VStt$\vert$ $\varnothing$    }

    \end{tabular}&
    \visible<2->{\begin{tabular}{lr}
      VS\phantom{$\vert$}$\to $ var\\
      \phantom{VStt}$\vert$ $\lambda $var.VS\\
      \phantom{VStt}$\vert$ (VS VS)\\
      \phantom{VStt}$\vert$ primitive    \\
      \phantom{VStt}$\vert$ VS$\uplus$VS&\emph{nondeterministic choice}    \\
      \phantom{VStt}$\vert$ $\Lambda$&\emph{choose any expression}    \\
      \phantom{VStt}$\vert$ $\varnothing$    &\emph{choose no expression}
    \end{tabular}}
  \end{tabular}

  
  \visible<3>{what version spaces mean:\begin{align*}
      \denotation{\text{var}}& = \left\{\text{var} \right\}&
      \denotation{v_1 \uplus v_2}& = \left\{e : v\in \left\{v_1,v_2 \right\},\;e\in \denotation{v} \right\}\\
      \denotation{\lambda x. v}& = \left\{\lambda x. e : e\in \denotation{v} \right\}&    
      \denotation{(v_1\; v_2)}& = \left\{(e_1\;e_2) : e_1\in \denotation{v_1},\;e_2\in \denotation{v_2} \right\}\\
      \denotation{\varnothing}& = \varnothing&
      \denotation{\Lambda}& = \Lambda
  \end{align*}}

\end{frame}

\begin{frame}{Using version spaces}
  \begin{tabular}{lr}
    \multicolumn{2}{l}{VS\phantom{$\vert$}$\to $ var\phantom{t}$\vert$ $\lambda $var.VS\phantom{t}$\vert$ (VS VS)\phantom{t}$\vert$ primitive}    \\
    \phantom{VStt}$\vert$ VS$\uplus$VS&\emph{nondeterministic choice}    \\
    \phantom{VStt}$\vert$ $\Lambda$&\emph{choose any expression}    \\
    \phantom{VStt}$\vert$ $\varnothing$    &\emph{choose no expression}
  \end{tabular}

  \vfill
  
\only<2>{exploit the fact that $e\in \denotation{v}$ can be efficiently computed:  
  \begin{align*}
  \textsc{refactor}(v|\text{Lib}) &= \begin{cases}
    e\text{, if $e\in \text{Lib}$ and $e\in \denotation{v}$}\\
    \textsc{refactor}'(v|\text{Lib})\text{, otherwise.}
  \end{cases}
\end{align*}
\begin{align*}
  \textsc{refactor}'(v|\text{Lib})& = v\text{, if $v$ is a primitive or variable}\\
  \textsc{refactor}'(\lambda x.b|\text{Lib}) &= \lambda x. \textsc{refactor}(b|\text{Lib})\\
  \textsc{refactor}'(v_1\;v_2|\text{Lib}) &= (\textsc{refactor}(v_1|\text{Lib})\phantom{t}\textsc{refactor}(v_2|\text{Lib}))\\
%  \textsc{refactor}'(v_1 \uplus v_2|\text{Lib}) &= \argmin_{e\in \left\{\textsc{refactor}(v|\text{Lib})\;:\;v\in V \right\}}\text{size}(e|\text{Lib})
  \end{align*}}

\only<3->{%% invert $\beta$-reduction via new $I\beta$ operator
  \centering

  \begin{tikzpicture}[every node/.style={inner sep=1,outer sep=0,rounded corners,thick}]
    \node[draw, rounded corners](u1) at (0,0) {union, $\uplus$};
    \node[draw, rounded corners](u2) at ($(0,-1) + (u1.south)$) {\code{(}$\uplus$\code{ 1)}}; \draw (u1.south) -- (u2.north);
    \node[draw, rounded corners](u21) at ($(-1.3,-1) + (u2.south)$) {\code{(($\lambda$ (x) (x 1)) +)}}; \draw ([xshift=-0.2cm]u2.south) -- (u21.north);
    \node[draw, rounded corners](u22) at ($(2.3,-1) + (u2.south)$) {\code{(($\lambda$ (x) (+ x)) 1)}};  \draw ([xshift=-0.2cm]u2.south) -- (u22.north);

    \node[draw, rounded corners](u11) at ($(-1.75,-1) + (u1.south)$) {\code{(+ 1 1)}}; \draw (u1.south) -- (u11.north);
    \node[draw, rounded corners](u12) at ($(2.75,-1) + (u1.south)$) {\code{(($\lambda$ (x) (x 1 1)) +)}}; \draw (u1.south) -- (u12.north);

    \node(vs) at ($(u21.west)!0.5!(u12.east) + (0,-1.25)$) {$\underbrace{\hspace{8cm}}_{\text{\normalsize Subset of version space}}$};

    \node[anchor=north](p) at ($(0,2.5) + (u1)$) {\normalsize Program: \texttt{(+ 1 1)}};
    \draw[ultra thick,->] ($(0,-0.1) + (p.south)$) --node[sloped, above, inner sep=5]{$I\beta$} ($(0,0.1) + (u1.north)$);  
\end{tikzpicture}
}

\only<4>{
  \messageOverlay{\textbf{completeness:} $I\beta$ gets all the refactorings\\
    let $v_2 = I\beta(v_1)$ and $e_1\in \denotation{v_1}$. for any $e_2\reduce e_1$ then $e_2\in \denotation{v_2}$\\\\
    \textbf{consistency:} $I\beta$ only gets valid refactorings\\
    let $v_2 = I\beta(v_1)$ and $e_2\in \denotation{v_2}$. then there is a $e_1\in \denotation{v_1}$ where $e_2\reduce e_1$
  }
  }

\end{frame}
\end{comment}


\begin{frame}{DreamCoder Domains}
  \only<1>{\includegraphics[width = \textwidth]{statement/taskbar2.png}}
  \only<2>{\includegraphics[width = \textwidth]{statement/taskbar3.png}}
\end{frame}

\begin{frame}{LOGO Graphics}
  30 out of 160 tasks
  \includegraphics[width = \textwidth]{dc/logoTasks30.png}
\end{frame}

\begin{frame}{LOGO Graphics -- learning interpretable library of concepts}
  \Wider[5em]{
    \includegraphics[width = \textwidth]{dc/logo_kathy.png}
  }

  \only<2>{
  \messageOverlay{
    \begin{tabular}{rl}
    circle$(r)$
    &\raisebox{-.5\height}{\includegraphics[width = 0.3\textwidth]{dc/logo_primitives/circle_negative.png}}\\
    polygon$(n,\ell)$
    &\raisebox{-.5\height}{\includegraphics[width = 0.3\textwidth]{dc/logo_primitives/polygon_negative.png}}
  \end{tabular}
  }}
  \only<3>{
    \messageOverlay{\begin{tabular}{c}
    radial symmetry$(n,\text{body})$\\
    \includegraphics[width = 0.35\textwidth]{dc/rotationalmontage_negative.png}
  \end{tabular}}
    }
\end{frame}

\begin{frame}{what does DreamCoder dream of?}
  \Wider[5em]{
    \only<1>{\begin{tabular}{lll}
    before learning&&after learning\\
    \includegraphics[height=3cm]{dc/dreams/beforeLearning25September11}&&
    \includegraphics[height=3cm]{dc/dreams/cherry_picked/montageSeptember14}    
    \end{tabular}}
    %% \only<2>{\begin{tabular}{lll}
    %%     before learning&&after learning\\
    %%     \includegraphics[width = 0.4\textwidth]{dc/dreams/appendixlogoinitial.png}&&
    %%     \includegraphics[width = 0.4\textwidth]{dc/dreams/appendixlogofinal.png}
    %%     \end{tabular}}
    }

\end{frame}

\begin{frame}{Planning to build towers}
  \Wider[5.5em]{
    \footnotesize
    \begin{tabular}{l}
      {example tasks (112 total)}\\
      \includegraphics[clip, trim = 0 0cm 0 2.9cm,width = 0.9\textwidth]{dc/tower_montage_21_negative.png}\\\\

    \end{tabular}
    \pause
    \begin{tabular}{l}
      {learned library routines ($\approx $ 20 total)}\\
      \begin{tabular}[t]{rlrl}
        arch$(h)$&%\begin{tabular}{l}
        \raisebox{-.1\height}{\includegraphics[width = 0.25\textwidth]{dc/tower/tower_dsl_towerArch.png}}&
        %\end{tabular}&
        pyramid$(h)$&
        \raisebox{-.1\height}{\includegraphics[width = 0.25\textwidth]{dc/tower/tower_dsl_pyramid.png}}
        \\
        wall$(w,h)$&
        \raisebox{-.3\height}{\includegraphics[width = 0.25\textwidth]{dc/tower/tower_dsl_bricks.png}}
        &%\\\\
        %% stairs$(h)$&\begin{tabular}{l}
        %%   \includegraphics[width = 0.25\textwidth]{dc/tower/tower_dsl_staircase.png}
        %% \end{tabular}\\\\
        \phantom{bbb}bridge$(w,h)$&
        \raisebox{-.3\height}{\includegraphics[width = 0.25\textwidth]{dc/tower/tower_dsl_bridge.png}}
        %\\\\\\
      \end{tabular}\\\\

  \end{tabular}}

  \pause

  \messageOverlay{
    \begin{tabular}{ll}
          \textbf{dreams before learning}&\textbf{dreams after learning}\\
          \begin{tabular}{c}
            \includegraphics[clip,trim = 0 4.5cm 0 0cm,width = 0.4\textwidth]{dc/tower/dreams/cherry_montage_initial_20.png}
          \end{tabular}\phantom{testing}&
          \begin{tabular}{c}
            \includegraphics[clip,trim = 0 4.5cm 0 0cm,width = 0.4\textwidth]{dc/tower/dreams/cherry_montage_final.png}
            \end{tabular}
          \\\\
      \end{tabular}
  }
  
\end{frame}

\begin{comment}
  \begin{frame}{Synergy between dreaming and library learning}
    \begin{tabular}{ll}
      %    \textbf{A}\\
      \only<1>{\includegraphics[height = 2cm]{../dreamcoder/figures/learningCurves/revision/text_hits_average_pretty_small_yl_stage1.png}}
      \only<2>{\includegraphics[height = 2cm]{../dreamcoder/figures/learningCurves/revision/text_hits_average_pretty_small_yl_stage2.png}}
      \only<3>{\includegraphics[height = 2cm]{../dreamcoder/figures/learningCurves/revision/text_hits_average_pretty_small_yl_stage3.png}}
      %      \only<4->{\includegraphics[height = 2cm]{../dreamcoder/figures/learningCurves/revision/text_hits_average_pretty_small_yl.png}}
      &
      \phantom{ttt}%
      \only<1>{\includegraphics[height = 2cm]{../dreamcoder/figures/learningCurves/revision/logo_hits_average_pretty_small_stage1.png}}
      \only<2>{\includegraphics[height = 2cm]{../dreamcoder/figures/learningCurves/revision/logo_hits_average_pretty_small_stage2.png}}
      \only<3>{\includegraphics[height = 2cm]{../dreamcoder/figures/learningCurves/revision/logo_hits_average_pretty_small_stage3.png}}\\
      %      \only<4->{\includegraphics[height = 2cm]{../dreamcoder/figures/learningCurves/revision/logo_hits_average_pretty_small.png}}\\
      \only<1>{\includegraphics[height = 2cm]{../dreamcoder/figures/learningCurves/revision/list_hard_hits_average_pretty_small_yl_stage1.png}}
      \only<2>{\includegraphics[height = 2cm]{../dreamcoder/figures/learningCurves/revision/list_hard_hits_average_pretty_small_yl_stage2.png}}
      \only<3>{\includegraphics[height = 2cm]{../dreamcoder/figures/learningCurves/revision/list_hard_hits_average_pretty_small_yl_stage3.png}}&
      %      \only<4->{\includegraphics[height = 2cm]{../dreamcoder/figures/learningCurves/revision/list_hard_hits_average_pretty_small_yl.png}}&
      \phantom{ttt}%
      \only<1>{\includegraphics[height = 2cm]{../dreamcoder/figures/learningCurves/revision/rational_hits_average_pretty_small_stage1.png}}
      \only<2>{\includegraphics[height = 2cm]{../dreamcoder/figures/learningCurves/revision/rational_hits_average_pretty_small_stage2.png}}
      \only<3>{\includegraphics[height = 2cm]{../dreamcoder/figures/learningCurves/revision/rational_hits_average_pretty_small_stage3.png}}&
      %      \only<4->{\includegraphics[height = 2cm]{../dreamcoder/figures/learningCurves/revision/rational_hits_average_pretty_small.png}}\\
      \only<1>{\includegraphics[height = 2.29cm]{../dreamcoder/figures/learningCurves/revision/tower_hits_ws_average_pretty_small_yl_stage1.png}}
      \only<2>{\includegraphics[height = 2.29cm]{../dreamcoder/figures/learningCurves/revision/tower_hits_ws_average_pretty_small_yl_stage2.png}}
      \only<3>{\includegraphics[height = 2.29cm]{../dreamcoder/figures/learningCurves/revision/tower_hits_ws_average_pretty_small_yl_stage3.png}}&
      %      \only<4->{\includegraphics[height = 2.29cm]{../dreamcoder/figures/learningCurves/revision/tower_hits_ws_average_pretty_small_yl.png}}&
      \phantom{tt.}%
      \only<1>{\includegraphics[height = 2.29cm]{../dreamcoder/figures/learningCurves/revision/regex_marginal_test_unigram_gen_ws_stage1.png}}
      \only<2>{\includegraphics[height = 2.29cm]{../dreamcoder/figures/learningCurves/revision/regex_marginal_test_unigram_gen_ws_stage2.png}}
      \only<3>{\includegraphics[height = 2.29cm]{../dreamcoder/figures/learningCurves/revision/regex_marginal_test_unigram_gen_ws_stage3.png}}
      %      \only<4->{\includegraphics[height = 2.29cm]{../dreamcoder/figures/learningCurves/revision/regex_marginal_test_unigram_gen_ws.png}}%
      %    \includegraphics[height = 2.29cm]{../dreamcoder/figures/learningCurves/revision/regex_marginal_test_unigram_gen_ws.png}%
      \phantom{tt}\includegraphics[width = 2.5cm]{../dreamcoder/figures/learningCurves/revision/curveLegend.png}\hspace{-3cm}
      %    \includegraphics[width = 0.25\textwidth]{../dreamcoder/figures/learningCurves/revision/depthVersusAccuracy_revision_MAX.png}
    \end{tabular}


    %% }
    %% }

  \end{frame}
\end{comment}

\begin{frame}{Synergy between dreaming and library learning}
  \begin{tabular}{lr}
    \begin{tabular}{l}
    \only<1>{\includegraphics[height = 2.29cm]{../dreamcoder/figures/learningCurves/revision/tower_hits_ws_average_pretty_small_yl_stage1.png}}
    \only<2>{\includegraphics[height = 2.29cm]{../dreamcoder/figures/learningCurves/revision/tower_hits_ws_average_pretty_small_yl_stage2.png}}
    \only<3>{\includegraphics[height = 2.29cm]{../dreamcoder/figures/learningCurves/revision/tower_hits_ws_average_pretty_small_yl_stage3.png}}
  \end{tabular}

    &

    \begin{tabular}{l}
    \only<2->{\includegraphics[width = 3cm]{../dreamcoder/figures/learningCurves/revision/curveLegend.png}}
  \end{tabular}
    \end{tabular}




  %% }
  %% }

\end{frame}

\begin{frame}{Synergy between dreaming and library learning}
  \begin{tabular}{ll}
    \includegraphics[height = 2cm]{../dreamcoder/figures/learningCurves/revision/text_hits_average_pretty_small_yl_stage3.png}
    %      \only<4->{\includegraphics[height = 2cm]{../dreamcoder/figures/learningCurves/revision/text_hits_average_pretty_small_yl.png}}
    &
    \phantom{ttt}%
    \includegraphics[height = 2cm]{../dreamcoder/figures/learningCurves/revision/logo_hits_average_pretty_small_stage3.png}\\
    \includegraphics[height = 2cm]{../dreamcoder/figures/learningCurves/revision/tower_hits_average_pretty_small_yl_stage3.png}&
    \phantom{ttt}%
    \includegraphics[height = 2cm]{../dreamcoder/figures/learningCurves/revision/rational_hits_average_pretty_small_stage3.png}\\
    \includegraphics[height = 2cm]{../dreamcoder/figures/learningCurves/revision/list_hard_hits_average_pretty_small_yl_stage3.png}
    &
    \phantom{tt.}%
    \includegraphics[height = 2.29cm]{../dreamcoder/figures/learningCurves/revision/regex_marginal_test_unigram_gen_ws_stage3.png}
    %      \only<4->{\includegraphics[height = 2.29cm]{../dreamcoder/figures/learningCurves/revision/regex_marginal_test_unigram_gen_ws.png}}%
    %    \includegraphics[height = 2.29cm]{../dreamcoder/figures/learningCurves/revision/regex_marginal_test_unigram_gen_ws.png}%
    \phantom{tt}\includegraphics[width = 2.5cm]{../dreamcoder/figures/learningCurves/revision/curveLegend.png}\hspace{-3cm}
  \end{tabular}


  %% }
  %% }

\end{frame}

\begin{frame}{synergy between dreaming and library learning}
  \begin{tikzpicture}[scale=1.5]
    {
    \begin{scope}[shift = {(1,-1)}]
    \node[align = center](synthesis) at (6,4) {Problem-solving};
    \node[align = center](Library) at (3,1) {Library};
    \node[align = center](recognitionModel) at (9,1) {Recognition \\model};

    \draw [->,thick] (synthesis.-120) to[out = -150,in = 60] node[below,rotate = 45,align = center]{{\footnotesize Trains}\\{\footnotesize (Abstraction)}} (Library.30);
    \draw [->,thick] (synthesis.-60) to[out = -30,in = 120] node[below,rotate=-45,align = center]{{\footnotesize Trains}\\{\footnotesize (Dreaming)}} (recognitionModel.150);
    \draw [->,thick] (Library.east) to[out = -30,in = 210] node[above, align = center]{{\footnotesize Trains}\\{\footnotesize (Dreaming)}} (recognitionModel.west);

    \draw [->,thick,dashed] (Library.north) to[out = 90,in = 180] node[fill=white,inner sep=0pt,align = center]{  \footnotesize{Inductive bias}\\\footnotesize{Hypothesis space}\\\footnotesize{(Wake)}} (synthesis.west);
    \draw [->,thick,dashed] (recognitionModel.north) to[out = 90,in = 0] node[fill=white,inner sep=0pt,align = center]{{\footnotesize Makes tractable}\\{\footnotesize (Wake)}} (synthesis.east);
  \end{scope}}
    \end{tikzpicture}

\end{frame}

\begin{frame}{Evidence for dreaming bootstrapping better libraries}
    \begin{tabular}{rr}
%        \multicolumn{1}{l}{\textbf{B}}\\
    \includegraphics[height = 0.3\textwidth]{../dreamcoder/figures/learningCurves/revision/depthVersusAccuracy_revision_MEAN.png}&
    \visible<2>{\includegraphics[height = 0.3\textwidth]{../dreamcoder/figures/learningCurves/revision/depthVersusAccuracy_revision_SIZE.png}} \\\\
\multicolumn{2}{c}{    \includegraphics[height = 1cm]{../dreamcoder/figures/learningCurves/revision/scatterLegend.png}}
    \end{tabular}
    Darker: Early in learning

    Lighter: Later in learning

\end{frame}

\begin{frame}{Variability in learned library}
  \Wider[5em]{\includegraphics[width = \textwidth]{deepArray.pdf}}

  \end{frame}

\begin{frame}{From learning libraries to learning languages}

  \huge

  How far can we push library learning?
  
  %% these experiments study how DreamCoder grows from a ``beginner'' state to ``expert'':
  %% \begin{itemize}
  %% \item ``beginner:'' basic domain-specific procedures, only easiest problems have short solutions
  %%   \item ``expert:'' learned library allowing hardest problems to have short meaningful solutions
  %%   \end{itemize}
  %% \pause
  %% go beyond: start with generic arithmetic \& control flow, learn fundamentals of domain

  %% \begin{itemize}
  %% \item Physics from recursive higher-order functions
  %%   \item Recursive higher-order language from 1959 proto-Lisp \& Y-combinator
  %% \end{itemize}
\end{frame}

\begin{frame}{Rediscovering vector algebra}
\Wider[5em]{  \includegraphics[width = 0.97\textwidth]{dc/physics_only.png}}
\end{frame}

\begin{frame}{Rediscovering origami programming}
  \Wider[5em]{  \includegraphics[width = \textwidth]{dc/mccarthy_only.png}}
\end{frame}

\begin{frame}{Lessons}

  Symbols aren't necessarily interpretable. Flexibly grow the language based on experience to make it more powerful \emph{and} more human understandable

  \vspace{1cm}

  Learning-from-scratch is possible in principle. Don't do it. Choose wisely what to learn and what to build in
\end{frame}

\begin{frame}{}
  \begin{center}
    \begin{tabular}{l}
      {\textcolor{black}{Program Induction and }\textcolor{gray}{perception}}\\
      \phantom{Program Induction and }{\textcolor{gray}{learning to learn}}\\
      \phantom{Program Induction and }{\textcolor{black}{interpretable models}}
      \end{tabular}
  \end{center}
\end{frame}

\begin{frame}[fragile]{Synthesizing human-understandable models of language}
  \begin{center}
    \begin{tikzpicture}[scale=0.75]
      \input{phonology/languageWheel50.tex}
    \end{tikzpicture}
  \end{center}
  \pause
  \messageOverlay{\large many languages, 70 diverse benchmarks\\\\
\large children and linguists can learn from sparse data\\\\
\large  linguists can communicate their knowledge}
\end{frame}

%% \begin{frame}{Few-shot language learning experiment}
%%   Farsi:

%%   \vspace{1cm}
  
%%   \begin{tabular}{rll}
%%     &\multicolumn{1}{l}{singular}&\multicolumn{1}{l}{plural}\\
%%       ``lip''& \phantom{tt}\textipa{l\ae b}&\phantom{tt}\textipa{l\ae ban}\\
%%       ``gazelle''& \phantom{tt}\textipa{ahu}& \phantom{tt}\textipa{ahuan}\\
%%        ``mother''& \phantom{tt}\textipa{valede}& \only<1>{\phantom{tt}???}\only<2->{\phantom{tt}\textipa{valedean}}
%%     \end{tabular}
%% %  \end{tabular}

%% %  \only<6>{
%% %  \messageOverly{\textbf{morphology} (subparts of words) + \textbf{phonology} (pronunciation)}}

%% \end{frame}

\begin{frame}{Few-shot language learning experiment}
  Mandarin:

  \vspace{1cm}

  \begin{tabular}{rcc}
    & adjective & adverb\\
    ``slow'' & \textipa{man} & \textipa{manmand@}\\
    ``fast'' & \textipa{kuai} & \textipa{kuaikuaid@}\\
    ``small'' & \textipa{xiao} & \only<2->{\textipa{xiaoxiaod@}}\only<1>{???}
  \end{tabular}

  \vspace{1cm}
  
  \only<3>{stem+stem+\textipa{d@}}

  \end{frame}

\begin{frame}{Few-shot language learning experiment}
  Serbo-Croatian:
  \vspace{1cm}
  
  \begin{tabular}{rlll}
    &\multicolumn{1}{c}{masculine}&\multicolumn{1}{c}{feminine}&\only<5->{stem (unobserved)}\\
    ``rich''&\textipa{bogat}&\textipa{bogata}&\only<5->{bogat}\\
    ``mild''&\textipa{blag}&\textipa{blaga}&\only<5->{blag}\\
    ``green''&\textipa{zelen}&\only<1>{???}\only<2->{\textipa{zelena}}&\only<5->{zelen}\\
    \only<4->{``clear''}&\only<4>{???}\only<5>{\textbf{yasan}}& \only<4->{yasna}&\only<5->{yasn}\\  
  \end{tabular}

  \vspace{1cm}
  
  \only<3->{\emph{add ``a'' to stem to make feminine}}
  
  \only<5->{
    \emph{insert ``a'' between two word-final consonants}\\
    \phonb{$\varnothing$}{a}{C}{C\#}
  }

\end{frame}


%% \begin{frame}{Diverse Linguistic Phenomena}
%%   \includegraphics[width = \textwidth]{phonology/Croatian}
%% \end{frame}
%% \begin{frame}{Languages with tones}
%%   \includegraphics[width = \textwidth]{phonology/tonal}
%% \end{frame}
\begin{frame}
  \begin{center}
    \begin{tabular}{ccc}
      \includegraphics[width = 0.3\textwidth]{figures/odden}&
      \includegraphics[width = 0.3\textwidth]{figures/spe}&
      \includegraphics[width = 0.3\textwidth]{figures/roca}
    \end{tabular}
  \end{center}
\end{frame}
\begin{frame}{}
\Wider[5em]{  \includegraphics[width = \textwidth]{figures/Turkish1}}
\end{frame}
\begin{frame}{Turkic Sakha (Yakut)}
\newcommand{\R}[1]{$\xrightarrow{r_{#1}}$}
\newcommand{\li}[2]{$\text{\textsc{#2}}: \text{\textipa{#1}}$}

\usetikzlibrary{calc}

\definecolor{observationColor}{RGB}{204,51,0}
\definecolor{languageColor}{RGB}{0,128,102}
\definecolor{lexiconColor}{RGB}{51,76,128}
\definecolor{universalColor}{RGB}{255,128,128}

\newcommand{\surfaceTriple}[4]{\phantom{t}\textcolor{observationColor}{\textsc{#1}}&\surface{#2}&\surface{#3}\underline{\surface{#4}}}

\newcommand{\ip}[1]{\textipa{#1}}
\newcommand{\surface}[1]{{\color{observationColor}\textipa{#1}}}

\newcommand{\meaning}[1]{ {\color{observationColor} #1} }

\newcommand{\chain}[3]{ \phantom{1}\vphantom{\R{1}}{\color{observationColor}\textsc{#1}}$\to$#2{\color{observationColor}\textipa{#3}}}
\renewcommand{\chain}[3]{ \phantom{1}\vphantom{\R{1}}{\color{observationColor}\textsc{#1}}:\phantom{tt}{\color{observationColor}\textipa{#3}}\phantom{test}}
\newcommand{\chainml}[4]{ \phantom{1}\vphantom{\R{1}}%
  {\color{observationColor}\textsc{#1}}$\to$#2\\
  \phantom{1}\vphantom{\R{1}}\phantom{testing}#3{\color{observationColor}[\textipa{#4}]}}

\renewcommand{\arraystretch}{0.5}
\setlength{\tabcolsep}{0.0em} % for the horizontal padding


\newcommand{\qualitativeLong}[7]{
    \begin{tikzpicture}[every node/.style={inner sep=1}]
      
      \visible<4->{
        \node(l)[anchor=north west,align=left,draw] at (0,0) {
          \begin{tabular}{r}
            #3
          \end{tabular}
        };
        \node(lexiconLabel)[anchor=south,align=center] at (l.north) {\large stems\\\large (unobserved)};
      }

    \node(d)[anchor=north west,draw=observationColor] at ([xshift=0.5cm]l.north east) {
      \begin{tabular}{l}
        #4
      \end{tabular}
    };
    \node(observationLevel)[anchor=south west,align=left] at (d.north -| d.west) {\large{\color{observationColor}observed data}};
    \visible<2->{
      \node[align=left,draw,anchor=south west](t) at ($(d.north west)+(0,1)$) {
      \begin{tabular}{l}
        #2
      \end{tabular}
      };
      \node[anchor= south ] at (t.north) {\large grammar (unobserved)};
      \draw[very thick,->] ([xshift=0.2cm]observationLevel.east |- t.south) -- ([xshift=0.2cm]d.north -| observationLevel.east);
    }
%    \node[anchor=south](language) at (t.north) {\large\textbf{#1}};

    




%    \draw[color=white] (lexiconLabel.north west) -- (t.north west) node[midway,sloped,color=black]{\Large\textbf{#1}};

    \visible<4->{
      \foreach \n in {0,...,#5}
               {
                 \draw[->] ($(l.north east)-(0.1,0.45)+\n*(0,-0.3)$) -- ($(d.north west)-(-0.1,0.45)+\n*(0,-0.3)$);
                 %-0.5*(0,\n)
               }
    }
  \end{tikzpicture}
}



\newcommand{\success}{
  \qualitativeLong{Turkic Sakha (Yakut)}{
    %    	# Nom_sg 	Gen_sg 		Nom_pl 		Gen_pl 		gloss
    \textsc{Singular}$\to $stem\\
    \textsc{Plural}$\to $stem+\ip{lar}\\
%    \textsc{Associative}$\to $stem+\ip{l11n}\\
    \\
\visible<3->{    $r_1$: \phonl{\textipa{l}}{\textipa{d}}{[ -lateral -tense ]}\\
    \phantom{$r_1$: }\textbf{``l'' becomes ``d'' next to ``r'', ``t'', but not ``l''}\\\\
    $r_2$: \phonl{C}{[ -voice ]}{[ -voice ]}\\
    \phantom{$r_1$: }\textbf{do not voice next to voiceless}\\\\
    $r_3$: \phonl{V}{[ +rounded ]}{[ +rounded ] [ -low ]\textsubscript{0}}\\
    $r_4$: \phonl{[ +continuant -high ]}{[ -rounded ]}{\textipa{u} C\textsubscript{0}}\\
    \phantom{$r_1$: }\textbf{``harmonize'' round vowels like ``u'', ``o''}\\\\
    $r_5$: \phonl{V}{[ -back -low ]}{[ -back +vowel ] [ ]\textsubscript{0}}\\
    \phantom{$r_1$: }\textbf{``harmonize'' vowels to be not at back of mouth}\\\\
    $r_6$: \phonl{[ -sonorant +voice ]}{[ +nasal ]}{[ +nasal ]}\\
    \phantom{$r_1$: }\textbf{``nasalize'' consonant next to a nasal, like ``m''}
}
  }{
    \\\addlinespace[0.1cm]
    \li{oron}{Bed}\\
    \li{bie}{Mare}\\
    \li{1skaap}{Cabinet}\\
    %\li{\"orus}{river}\\
    \addlinespace[0.15cm]\\
  }{
    %    \chain{faces (nom)}{\ip{j\"uz}+\ip{lar}$\to $\ip{j\"uzlar}\R{1}}{j\"uzler}\\
    %% \begin{comment}
    %% \chain{bed (singular)}{t}{\phantom{testt}oron}\phantom{te}\chain{bed (plural)}{\ip{oron}+\ip{lar}$\to $\ip{oronlar}\R{1}\ip{orondar}\R{3}\ip{orondor}\R{6}}{\phantom{testt}oronnor}\\
    %% \chain{mare (singular)}{\phantom{test}t4}{\phantom{tes}bie}\phantom{test}\chain{mare (plural)}{\ip{bie}+\ip{lar}$\to $\ip{bielar}\R{5}}{\phantom{tes}bieler}\\
    %% \chain{cabinet (singular)}{t}{1skaap}\chain{cabinet (plural)}{\ip{1skaap}+\ip{lar}$\to $\ip{1skaaplar}\R{1}\ip{1skaapdar}\R{2}}{1skaaptar}\\
    %% \end{comment}
    \begin{tabular}{lll}
      &\textcolor{observationColor}{\textsc{Singular}}\phantom{test}&\textcolor{observationColor}{\textsc{Plural}}\\\addlinespace[0.1cm]
      \surfaceTriple{Bed}{oron}{oron}{nor}\\
      \surfaceTriple{Mare}{bie}{bie}{ler}\\
      \surfaceTriple{Cabinet\phantom{te}}{1skaap}{1skaap}{tar}\\\addlinespace[0.1cm]
      \end{tabular}
\\    \multicolumn{1}{c}{\emph{138 total examples}}\\
    %% \chainml{river (assoc)}{\ip{\"orus}+\ip{l11n}$\to $\ip{\"orusl11n}\R{1}\ip{\"orusd11n}\R{2}}{%
    %%   \ip{\"orust11n}\R{3}\ip{\"orustuun}\R{5}}{\"or\"ust\"u\"un}
%    \chain{stamps (gen)}{\ip{pul}+\ip{un}$\to $}{pulun}\phantom{\R{1}}
%    \chain{rope (gen)}{\ip{ip}+\ip{un}$\to $\ip{ipun}$\to $}{ipin}
      %% \textipa{ip} & \textipa{ipin} & \textipa{ipler} & \textipa{iplerin}\\
      %% \textipa{k1z} & \textipa{k1z1n} & \textipa{k1zlar} & \textipa{k1zlar1n}\\
      %% \textipa{j\"uz} & \textipa{j\"uz\"un} & \textipa{j\"uzler} & \textipa{j\"uzlerin}\\
      %% \textipa{pul} & \textipa{pulun} & \textipa{pullar} & \textipa{pullar1n}\\
      %% \textipa{el} & \textipa{elin} & \textipa{eller} & \textipa{ellerin}\\
      %% \textipa{t\super San} & \textipa{t\super San1n} & \textipa{t\super Sanlar} & \textipa{t\super Sanlar1n}\\
      %% \textipa{k\"oj} & \textipa{k\"oj\"un} & \textipa{k\"ojler} & \textipa{k\"ojlerin}\\
      %% \textipa{son} & \textipa{sonun} & \textipa{sonlar} & \textipa{sonlar1n}
  }{2}{138}{2}
}



\Wider[5em]{
  \begin{center}
    \scriptsize\success
    \end{center}}}
%  \includegraphics[width = \textwidth]{phonology/harmony}
\end{frame}
\begin{frame}{Turkic Sakha (Yakut)}
\newcommand{\R}[1]{$\xrightarrow{r_{#1}}$}
\newcommand{\li}[2]{$\text{\textsc{#2}}: \text{\textipa{#1}}$\vphantom{\R{1}$\rangle$}}

\usetikzlibrary{calc}

\definecolor{observationColor}{RGB}{204,51,0}
\definecolor{languageColor}{RGB}{0,128,102}
\definecolor{lexiconColor}{RGB}{51,76,128}
\definecolor{universalColor}{RGB}{255,128,128}

\newcommand{\ip}[1]{\textipa{#1}}
\newcommand{\surface}[1]{{\color{observationColor}\textipa{#1}}}

\newcommand{\meaning}[1]{ {\color{observationColor} #1} }

\newcommand{\chain}[3]{ \phantom{1}\vphantom{\R{1}}{\color{observationColor}\textsc{#1}}\visible<2->{$\to$}#2{\color{observationColor}\textipa{#3}}}
%\renewcommand{\chain}[3]{ \phantom{1}\vphantom{\R{1}}{\color{observationColor}\textsc{#1}}:\phantom{tt}{\color{observationColor}\textipa{#3}}\phantom{test}}
\newcommand{\chainml}[4]{ \phantom{1}\vphantom{\R{1}}%
  {\color{observationColor}\textsc{#1}}$\to$#2\\
  \phantom{1}\vphantom{\R{1}}\phantom{testing}#3{\color{observationColor}\textipa{#4}}}

\renewcommand{\arraystretch}{0.5}
\setlength{\tabcolsep}{0.0em} % for the horizontal padding


\newcommand{\qualitativeLong}[7]{
    \begin{tikzpicture}[every node/.style={inner sep=1}]
      
      {
        \node(l)[anchor=north west,align=left,draw] at (0,0) {
          \begin{tabular}{r}
            #3
          \end{tabular}
        };
        \node(lexiconLabel)[anchor=south,align=center] at (l.north) {\Large stems\\\large (unobserved)};
      }

    \node(d)[anchor=north west,draw=observationColor] at ([xshift=0.5cm]l.north east) {
      \begin{tabular}{l}
        #4
      \end{tabular}
    };
    \node(observationLevel)[anchor=south west,align=left] at (d.north -| d.west) {\Large{\color{observationColor}observed data}};
    {
      \node[align=left,draw,anchor=south west](t) at ($(d.north west)+(0,1)$) {
      \begin{tabular}{l}
        #2
      \end{tabular}
      };
      \node[anchor= south ] at (t.north) {\large grammar (unobserved)};
      \draw[very thick,->] ([xshift=0.2cm]observationLevel.east |- t.south) -- ([xshift=0.2cm]d.north -| observationLevel.east);
    }
%    \node[anchor=south](language) at (t.north) {\large\textbf{#1}};

    




%    \draw[color=white] (lexiconLabel.north west) -- (t.north west) node[midway,sloped,color=black]{\Large\textbf{#1}};

    {
      \foreach \n in {0,...,#5}
               {
                 \draw[->] ($(l.north east)-(0.1,0.3)+\n*(0,-0.38)$) -- ($(d.north west)-(-0.1,0.3)+\n*(0,-0.38)$);
                 %-0.5*(0,\n)
               }
    }
  \end{tikzpicture}
}



\newcommand{\success}{
  \qualitativeLong{Turkic Sakha (Yakut)}{
    %    	# Nom_sg 	Gen_sg 		Nom_pl 		Gen_pl 		gloss
    \textsc{Singular}$\to $stem\\
    \textsc{Plural}$\to $stem+\ip{lar}\\
%    \textsc{Associative}$\to $stem+\ip{l11n}\\
    \\
{\only<3>{$\mathbf{r_1}$}\only<4->{$r_1$}\only<1-2>{$r_1$}: \phonl{\textipa{l}}{\textipa{d}}{[ -lateral -tense ]}\\
    \phantom{$r_1$: }\textbf{``l'' becomes ``d'' next to ``r'', ``t'', but not ``l''}\\\\
    $r_2$: \phonl{C}{[ -voice ]}{[ -voice ]}\\
    \phantom{$r_1$: }\textbf{do not voice next to voiceless}\\\\
    \only<4>{$\mathbf{r_3}$}\only<5->{$r_3$}\only<1-3>{$r_3$}: \phonl{V}{[ +rounded ]}{[ +rounded ] [ -low ]\textsubscript{0}}\\
    $r_4$: \phonl{[ +continuant -high ]}{[ -rounded ]}{\textipa{u} C\textsubscript{0}}\\
    \phantom{$r_1$: }\textbf{``harmonize'' round vowels like ``u'', ``o''}\\\\
    $r_5$: \phonl{V}{[ -back -low ]}{[ -back +vowel ] [ ]\textsubscript{0}}\\
    \phantom{$r_1$: }\textbf{``harmonize'' vowels to be not at back of mouth}\\\\
    \only<5->{$\mathbf{r_6}$}%% \only<6->{$r_6$}
    \only<1-4>{$r_6$}: \phonl{[ -sonorant +voice ]}{[ +nasal ]}{[ +nasal ]}\\
    \phantom{$r_1$: }\textbf{``nasalize'' consonant next to a nasal, like ``m''}
}
  }{
    \li{oron}{bed}
    %% \visible<6->{\li{bie}{mare}}\\
    %% \visible<6->{\li{1skaap}{cabinet}}\\
    %% \visible<6->{\li{\"orus}{river}}%% \\
    %% \phantom{\li{l}{l}}
  }{
%    \chain{faces (nom)}{\ip{j\"uz}+\ip{lar}$\to $\ip{j\"uzlar}\R{1}}{j\"uzler}\\
    \chain{beds}{\visible<2->{\ip{oron}+\ip{lar}$\to $\ip{oronlar}}\visible<3->{\R{1}\ip{orondar}}\visible<4->{\R{3}\ip{orondor}}\visible<5->{\R{6}}}{\visible<5->{oronnor}}
 %%    \visible<6->{\chain{mares}{\ip{bie}+\ip{lar}$\to $\ip{bielar}\R{5}}{bieler}}\\
 %%    \visible<6->{\chain{cabinets}{\ip{1skaap}+\ip{lar}$\to $\ip{1skaaplar}\R{1}\ip{1skaapdar}\R{2}}{1skaaptar}}\\
 %% \visible<6->{\chainml{river (assoc)}{\ip{\"orus}+\ip{l11n}$\to $\ip{\"orusl11n}\R{1}\ip{\"orusd11n}\R{2}}{%
 %%   \ip{\"orust11n}\R{3}\ip{\"orustuun}\R{5}}{\"or\"ust\"u\"un}}
%    \chain{stamps (gen)}{\ip{pul}+\ip{un}$\to $}{pulun}\phantom{\R{1}}
%    \chain{rope (gen)}{\ip{ip}+\ip{un}$\to $\ip{ipun}$\to $}{ipin}
      %% \textipa{ip} & \textipa{ipin} & \textipa{ipler} & \textipa{iplerin}\\
      %% \textipa{k1z} & \textipa{k1z1n} & \textipa{k1zlar} & \textipa{k1zlar1n}\\
      %% \textipa{j\"uz} & \textipa{j\"uz\"un} & \textipa{j\"uzler} & \textipa{j\"uzlerin}\\
      %% \textipa{pul} & \textipa{pulun} & \textipa{pullar} & \textipa{pullar1n}\\
      %% \textipa{el} & \textipa{elin} & \textipa{eller} & \textipa{ellerin}\\
      %% \textipa{t\super San} & \textipa{t\super San1n} & \textipa{t\super Sanlar} & \textipa{t\super Sanlar1n}\\
      %% \textipa{k\"oj} & \textipa{k\"oj\"un} & \textipa{k\"ojler} & \textipa{k\"ojlerin}\\
      %% \textipa{son} & \textipa{sonun} & \textipa{sonlar} & \textipa{sonlar1n}
  }{0}{138}{2}
}



\Wider[5em]{
  \begin{center}
    \scriptsize\success
    \end{center}}}
%  \includegraphics[width = \textwidth]{phonology/harmony}
\end{frame}


\begin{frame}{}
\Wider[5em]{  \includegraphics[width = \textwidth]{figures/languageMontage}}
\end{frame}
\begin{frame}[fragile]{Distilling higher-level knowledge}
  \Wider[5em]{
    \begin{center}
\begin{tikzpicture}[scale=2]
\input{languageHierarchy.tex}
\end{tikzpicture}
  \end{center}}
  \end{frame}
%% \begin{frame}{Distilling higher-level knowledge}
%%   \Wider[5em]{
%%     \begin{tabular}{rl}
%%       \begin{tabular}{c}
%%         Discovered\\ universal grammar \\schema
%%       \end{tabular}&
%%       \begin{tabular}{l}
%%         \multicolumn{1}{c}{\textbf{consonant/vowel distinction}}\\
%%         sounds$\gets$ [-vowel]\\
%%         sounds$\gets$ [+vowel]\\
%%         \emph{a set of sounds is commonly}\\\emph{ all consonants,}\\\emph{ or all vowels}
%%       \end{tabular}\\
%%       \begin{tabular}{c}
%%         w/o learned\\
%%         universal grammar
%%       \end{tabular}&
%%       Tibetan:\phantom{test}    \phonl{[-nasal]}{$\varnothing $}{\#}\\\\
%%       \begin{tabular}{c}
%%         w/ learned\\
%%         universal grammar
%%       \end{tabular}&
%%       \phantom{Tibetan:test} \phonb{[-vowel]}{$\varnothing $}{\# }{[-vowel]}

%% %% %    Kerewe&    \phonb{[ ]}{[ +hiTone ]}{[ +hiTone ] [ ] }{ [ ]}
%% %%   \end{tabular}&
      
%%     \end{tabular}
%% %%     \begin{tabular}{cll}
%% %%     \toprule&
%% %%   \multicolumn{1}{c}{Without learned universal grammar}&
%% %%   \multicolumn{1}{c}{With learned universal grammar}\\\midrule \\
%% %%   \begin{tabular}{l}
%% %%     \multicolumn{1}{c}{\textbf{consonant/vowel distinction}}\\
%% %%     sounds$\gets$ [-vowel]\\
%% %%     sounds$\gets$ [+vowel]\\
%% %%     \emph{a set of sounds is commonly}\\\emph{ all consonants,}\\\emph{ or all vowels}
%% %%   \end{tabular}&
%% %%   \begin{tabular}{rl}
%% %%     Tibetan&    \phonl{[-nasal]}{$\varnothing $}{\#}
%% %% %    Kerewe&    \phonb{[ ]}{[ +hiTone ]}{[ +hiTone ] [ ] }{ [ ]}
%% %%   \end{tabular}&
%% %%   \begin{tabular}{l}
%% %%     
%% %%   \end{tabular}
%% %%   \\\\\midrule   
%% %%   \begin{tabular}{l}
%% %%     \multicolumn{1}{c}{\textbf{nasal place assimilation}}\\
%% %%     $[\text{+nasal}]\to \alpha\text{place}$ / %\phonb{[+nasal]}{$\alpha$place}{\texttt{trigger}}{\phonfeat[c]{C\\$\alpha$place}}
%% %%     \underline{\phantom{aa}}\phonfeat[c]{C\\$\alpha$place}\\
%% %%     \emph{nasals (``n'', ``m'', ...) move to the place in the}\\
%% %%       \emph{ mouth where the next consonant is}
%% %%   \end{tabular}
%% %%   &
  
%% %%   \begin{tabular}{rl}    
%% %%     Bukusu&    \phonr{\textipa{\~n}}{$\alpha$place}{[$\alpha$place]}\\\\
%% %%   \end{tabular}
%% %%   &
%% %%   \begin{tabular}{l}
%% %%     \phonr{[+nasal]}{$\alpha$place}{\phonfeat[c]{C\\$\alpha$place}}\\\\
%% %%   \end{tabular}
%% %% \end{tabular}
%%   }
%% %\includegraphics[width = \textwidth]{phonology/universal}
%% \end{frame}




\begin{comment}
  \begin{frame}{From learning libraries to using interpreters}
    DreamCoder: library building as inspiration\\
    Software engineers: build libraries, use interpreters, version control, debuggers, ...

    \vfill
    \small  \underline{Ellis}$^*$, Nye$^*$, Pu$^*$, Sosa$^*$, Tenenbaum, Solar-Lezama. NeurIPS 2019. $^*$equal contribution

  \end{frame}

  \begin{frame}{Synthesis with a REPL}


    
    \Wider[5em]{  \includegraphics[width = \textwidth]{assets/wrench_wave}}

    REPL: Bridges syntax and semantics

    $\pi$: policy, writes code conditioned on REPL state

    $T$: Markov Decision Process transition function

    Spec: Image to draw

    \vfill
    \small  \underline{Ellis}$^*$, Nye$^*$, Pu$^*$, Sosa$^*$, Tenenbaum, Solar-Lezama. NeurIPS 2019. $^*$equal contribution
    
  \end{frame}

  %% \begin{frame}{Interleaving policy+value+REPL in stochastic tree search}
  %%   c.f. AlphaGo, TD-Gammon

  %%   \includegraphics[width = \textwidth]{assets/smc}

  %% \end{frame}

  \begin{frame}{Scaling to long programs}
    Branching factor: $ > 1.3$ million per line of code

    Successfully synthesizes $ > $20-line programs in seconds ($ > $100 tokens)

    \vspace{1cm}
    
    \Wider[5em]{  
      \setlength{\tabcolsep}{0pt}
      \renewcommand{\arraystretch}{0}  
      
      %    \includegraphics[width = \textwidth]{assets/pixel_montage} \\
      % \begin{tabular}{cc}
      %   \includegraphics[width = 0.3\textwidth]{assets/2d_synthetic_montage}&
      %   \includegraphics[width = 0.6\textwidth]{assets/tool_montage}
      % \end{tabular}\\\\    
      \begin{tabular}{cccccccc}
        \rotatebox[origin=l]{90}{Spec}
        &    \rotatebox[origin=l]{90}{(voxels)}$\;$&
        \includegraphics[width = 2cm]{assets/3-D/demo3/spec}&
        \includegraphics[width = 2cm]{assets/3-D/demo6/0000png_v.png}&
        \includegraphics[width = 2cm]{assets/3-D/demo2/spec}&
        \includegraphics[width = 2cm]{assets/3-D/demo4/012_png_v.png}&
        \includegraphics[width = 2cm]{assets/3-D/demo5/CAD_example_input}&
        \includegraphics[width = 2cm]{assets/3-D/demo1/spec}
        \\
        \rotatebox[origin=l]{90}{Rendered}&
        \rotatebox[origin=l]{90}{program}$\;$&
        \includegraphics[width = 2cm]{assets/3-D/demo3/000_SMC_value_pickle_pretty.png}$\;$&
        \includegraphics[width = 2cm]{assets/3-D/demo6/2}$\;$&
        \includegraphics[width = 2cm]{assets/3-D/demo2/pretty}$\;$&
        \includegraphics[width = 2cm]{assets/3-D/demo4/012_SMC_value_pickle_pretty}$\;$&
        \includegraphics[width = 2cm]{assets/3-D/demo5/CAD_example_output}$\;$&
        \includegraphics[width = 2cm]{assets/3-D/demo1/pretty}
      \end{tabular}
      \setlength{\tabcolsep}{6pt}
      \renewcommand{\arraystretch}{1}
    }
  \end{frame}
\end{comment}

\begin{frame}{Lessons}

  Higher-level knowledge matters (``universal grammar''). Get the basic computational substrate correct

  \vspace{1cm}

  But \emph{some} of this higher-level knowledge can be learned. You don't need millions of examples to learn it. But it's not a one-shot learning problem either

\end{frame}

\begin{frame}{}
  \begin{center}
    \begin{tabular}{l}
      {\textcolor{black}{Program Induction and }\textcolor{gray}{perception}}\\
      \phantom{Program Induction and }{\textcolor{gray}{learning to learn}}\\
      \phantom{Program Induction and }{\textcolor{gray}{interpretable models}}\\
      \phantom{Program Induction and }{\textcolor{black}{the future}}
      \end{tabular}
  \end{center}
\end{frame}

\begin{frame}{Models of the physical world}
  \begin{tabular}{rcc}
    \begin{tabular}{c}
      hinge
    \end{tabular}&
    \begin{tabular}{c}
      \includegraphics[width = 3cm]{figures/hinge}
    \end{tabular}&
    \begin{tabular}{c}
      \includegraphics[width = 3cm]{figures/laptopHinge}
    \end{tabular}\\
    \vspace{1cm}
    \\
    \multicolumn{1}{c}{
      \begin{tabular}{c}
        gear
      \end{tabular}
    }&
    %empty
    &
    \begin{tabular}{c}
      doorknob
    \end{tabular}\\
    \begin{tabular}{c}
      \includegraphics[width = 3cm]{figures/gears}
    \end{tabular}&
    &
    \begin{tabular}{c}
      \includegraphics[width = 3cm]{figures/doorknob}
    \end{tabular}
  \end{tabular}
\end{frame}

\begin{frame}{}
  \Wider[5em]{\includegraphics[width = \textwidth]{figures/futureGears}}

\end{frame}
\newcommand{\person}[2]{\begin{tabular}{c}
    {\footnotesize #1}\\
    \includegraphics[width = 1.7cm]{#2}
\end{tabular}}

\begin{frame}{Collaborators}
\Wider[4em]{  \begin{tabular}{cccc}
    \person{Tim O'Donnell}{collaborators/Timothy}&
    \person{Josh Tenenbaum}{collaborators/Josh}&
    \person{Adam Albright}{collaborators/Adam}&
    \person{Armando\\{\footnotesize Solar-Lezama}}{collaborators/Armando}\\
    \person{Max Nye}{collaborators/MaxNye}&
    \person{Cathy Wong}{collaborators/cw}&
    \person{Yewen Pu}{collaborators/Evan}&
    \person{Dan Ritchie}{collaborators/Daniel}\\
    \person{Mathias Sable-Meyer}{collaborators/French}&
    \person{Lucas Morales}{collaborators/Lucas}&
    \multicolumn{2}{c}{\textbf{\Large thank you}}
    \end{tabular}}
  

  \end{frame}
%% \begin{frame}{Scaling to long programs}
%%   Branching factor: $ > 400$ actions

%%   Successfully synthesizes 40-action programs

%%   \vspace{1cm}
  
%%   \Wider[5em]{
%% \centering
%%   \setlength{\tabcolsep}{2pt}
%%   \renewcommand{\arraystretch}{1}
%%   \footnotesize

%% \begin{tabular}{|lll|}
%% \cline{1-3}% \cline{5-6}
%% \multicolumn{3}{|c|}{Spec:}\\% &  & \multicolumn{3}{c|}{Spec:} \\
%% 6/12/2003 & $\to$ & date: 12 mo: 6 year: 2003\\% &  & Dr Mary Jane  Lennon & $\to$ & Lennon, Mary Jane (Dr) \\
%% 3/16/1997 & $\to$ & date: 16 mo: 3 year: 1997 \\%&  & Mrs Isaac  McCormick & $\to$ & McCormick, Isaac (Mrs) \\ \cline{1-3} \cline{5-7} 
%% \cline{1-3}\multicolumn{3}{|c|}{Held out test instance:}\\% &  & \multicolumn{3}{c|}{Held out test instance:} \\
%% 12/8/2019 & $\to$ & date: 8 mo: 12 year: 2019\\% &  & Dr James Watson & $\to$ & Watson, James (Dr) \\ \cline{1-3} \cline{5-7} 
%% \cline{1-3}\multicolumn{3}{|c|}{Results:}\\% &  & \multicolumn{3}{c|}{Results:} \\
%% \textbf{Ours} & $\to$ & \textbf{date: 8 mo: 12 year: 2019}\\% &  & \textbf{} & $\to$ & \textbf{Watson, James (Dr)} \\
%% %% Rollout & $\to$ & date: 8 mo: 1282019\\% &  &  & $\to$ & Watson, James \\
%% %% Beam w/value & $\to$ & date: 8 mo: 12 year:2019\\% &  &  & $\to$ & Watson, JamesDr \\
%% %% Beam & $\to$ & date: 8 mo: 12 year:\\% &  &  & $\to$ & Watson, James ( \\
%% RobustFill & $\to$ & date:12/8/2019\\% &  &  & $\to$ & Dr James  Watson \\ \cline{1-3} \cline{5-7}
%% \cline{1-3}
%% \end{tabular}
%%   \setlength{\tabcolsep}{6pt}
%%   \renewcommand{\arraystretch}{1}
%% }
%% \end{frame}



%% \begin{frame}{Vision}


%%    \underline{More human-like machine intelligence}\\%Flexibly adapting to new problem domains:
%%    \begin{itemize}
%%    \item    Acquiring a domain-specific representation (DSL)
%%      \item Learning  to use that representation (recognition model)
%%    \end{itemize}
%%    DreamCoder: an algorithm for jointly realizing these goals

   



%%    \hspace{-1cm}\includegraphics[width = 12cm]{ecFigures/finale.png}

%%    \pause

%%    \Huge \centering \textbf{The End.}
%%   \end{frame}

\end{document}
